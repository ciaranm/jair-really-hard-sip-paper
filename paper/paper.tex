% vim: set spell spelllang=en tw=100 :

\documentclass[twoside,11pt]{article}
\usepackage{jair}
\usepackage{theapa}
\usepackage{rawfonts}
\usepackage{complexity}
\usepackage{cleveref}
\usepackage{graphicx}
\usepackage{tikz}
\usepackage{amsmath}
\usepackage{amssymb}
\usepackage{nicefrac}
\usepackage{csquotes}

% \usepackage{showframe}

\usetikzlibrary{decorations, decorations.pathreplacing, calc, backgrounds}

\jairheading{1}{2017}{0-0}{0/0}{0/0}
\ShortHeadings{When Subgraph Isomorphism is Really Hard}
{McCreesh, Prosser, Solnon \& Trimble}
\firstpageno{1}

\definecolor{chromaygb1}{rgb}{1.0, 1.0, 0.878431}
\definecolor{chromaygb1b}{rgb}{0.8, 0.972549, 0.72549}
\definecolor{chromaygb2}{rgb}{0.560784,0.929412,0.564706}
\definecolor{chromaygb3}{rgb}{0.407843,0.776471,0.560784}
\definecolor{chromaygb4}{rgb}{0.227451,0.631373,0.552941}
\definecolor{chromaygb5}{rgb}{0.301961,0.431373,0.670588}
\definecolor{chromaygb6}{rgb}{0.317647,0.254902,0.588235}
\definecolor{chromaygb7}{rgb}{0.294118,0.0,0.509804}

\crefname{figure}{Figure}{Figures}
\Crefname{figure}{Figure}{Figures}
\crefname{section}{Section}{Sections}
\Crefname{section}{Section}{Sections}

\newcommand{\citet}[1]{\citeA{#1}}
\newcommand{\citep}[1]{\cite{#1}}

% http://tex.stackexchange.com/questions/22100/the-bar-and-overline-commands
\newcommand{\shortoverline}[1]{\mkern 1.5mu\overline{\mkern-1.5mu#1\mkern-1.5mu}\mkern 1.5mu}

\newcommand*\samethanks[1][\value{footnote}]{\footnotemark[#1]}

\begin{document}

\title{When Subgraph Isomorphism is Really Hard, and Why This Matters for Graph
Databases}

\author{\name Ciaran McCreesh \thanks{This work was supported by the Engineering and Physical Sciences
           Research Council [grant numbers EP/K503058/1, EP/M508056/1, and EP/P026842/1]} \email
           ciaran.mccreesh@glasgow.ac.uk \\
       \addr University of Glasgow, Glasgow, Scotland \\
       \name Patrick Prosser \samethanks[1] \email patrick.prosser@glasgow.ac.uk \\
       \addr University of Glasgow, Glasgow, Scotland \\
       \name Christine Solnon \thanks{This work was supported
       by the ANR project SoLStiCe (ANR-13-BS02-0002-01)} \email christine.solnon@insa-lyon.fr \\
       \addr INSA-Lyon, LIRIS, UMR5205, F-69621, France \\
       \name James Trimble \samethanks[1] \email j.trimble.1@research.gla.ac.uk \\
       \addr University of Glasgow, Glasgow, Scotland}
\maketitle

\begin{abstract}
    The subgraph isomorphism problem involves finding a copy of a pattern graph inside a larger
    target graph. The non-induced version allows extra edges in the target, whilst the induced
    version does not.  Although both variants are \NP-complete, algorithms inspired by constraint programming
    can operate comfortably on most real-world graphs with thousands of vertices.  However, they
    cannot handle arbitrary instances of this size. We show how to generate ``really hard'' random
    instances for subgraph isomorphism problems, which are computationally challenging with a couple
    of hundred vertices in the target, and only twenty pattern vertices. For the non-induced version
    of the problem, these instances lie on a satisfiable / unsatisfiable phase transition, whose
    location we can predict; for the induced variant, much richer behaviour is observed, and
    constrainedness gives a better measure of difficulty than does proximity to a phase transition.
    These results have practical consequences: we explain why the widely researched ``filter~/
    verify'' indexing technique used in graph databases is a flawed design (based upon a repeated
    misunderstanding of the empirical hardness of \NP-complete problems) that cannot be beneficial
    when paired with any reasonable subgraph isomorphism algorithm.
\end{abstract}

\section{Introduction}\label{section:introduction}

The \emph{non-induced subgraph isomorphism problem} is to find an injective mapping from the
vertices of a given pattern graph to the vertices of a given target graph which preserves
adjacency---in essence, we are ``finding a copy of'' the pattern inside the target. The
\emph{induced} variant of the problem additionally requires that the mapping preserve non-adjacency,
so there are no ``extra edges'' in the copy of the pattern that we find. We illustrate both variants
in \cref{figure:sip}.  Although these problems are \NP-complete, modern subgraph
isomorphism algorithms based upon constraint programming techniques can handle problem instances
with many hundreds of vertices in the pattern graph, and up to ten thousand vertices in the target
graph
\citep{DBLP:journals/ai/Solnon10,DBLP:conf/cp/AudemardLMGP14,DBLP:conf/cp/McCreeshP15,DBLP:conf/lion/KotthoffMS16},
and subgraph isomorphism is used successfully in application areas including computer vision
\citep{DBLP:journals/cviu/DamiandSHJS11,DBLP:journals/pr/SolnonDHJ15}, biochemistry
\citep{o:10.1371/journal.pone.0076911,DBLP:conf/gbrpr/CarlettiFV15}, and pattern recognition
\citep{DBLP:journals/ijprai/ConteFSV04}.  A labelled version of subgraph isomorphism
is also one of the key components in supporting complex queries in graph databases, which we discuss in
detail in \cref{section:filterverify}.

\begin{figure}[tb]
    \centering
    \begin{tikzpicture}[scale=0.3,every node/.style={font=\footnotesize}]
        \begin{scope}
            \node[draw, circle, fill=chromaygb2, inner sep=0.5pt] (Na) at (1,  0) {\phantom{0}};
            \node[draw, circle, fill=chromaygb2, inner sep=0.5pt] (Nb) at (1, -2) {\phantom{0}};
            \node[draw, circle, fill=chromaygb2, inner sep=0.5pt] (Nc) at (0, -4) {\phantom{0}};
            \node[draw, circle, fill=chromaygb2, inner sep=0.5pt] (Nd) at (2, -4) {\phantom{0}};

            \draw [ultra thick] (Na) -- (Nb);
            \draw [ultra thick] (Nb) -- (Nc);
            \draw [ultra thick] (Nc) -- (Nd);
            \draw [ultra thick] (Nb) -- (Nd);

            \node[draw, circle, fill=chromaygb2, inner sep=0.5pt] (N1) at (7.5,  0) {\phantom{0}};
            \node[draw, circle, fill=white, inner sep=0.5pt] (N2) at (9.5,  0) {\phantom{0}};
            \node[draw, circle, fill=chromaygb2, inner sep=0.5pt] (N3) at (7.5, -2) {\phantom{0}};
            \node[draw, circle, fill=white, inner sep=0.5pt] (N4) at (9.5, -2) {\phantom{0}};
            \node[draw, circle, fill=chromaygb2, inner sep=0.5pt] (N5) at (7.5, -4) {\phantom{0}};
            \node[draw, circle, fill=chromaygb2, inner sep=0.5pt] (N6) at (9.5, -4) {\phantom{0}};
            \node[draw, circle, fill=white, inner sep=0.5pt] (N7) at (5.5,  0) {\phantom{0}};
            \node[draw, circle, fill=white, inner sep=0.5pt] (N8) at (5.5, -2) {\phantom{0}};
            \node[draw, circle, fill=white, inner sep=0.5pt] (N9) at (5.5, -4) {\phantom{0}};

            \draw [] (N1) -- (N2);
            \draw [ultra thick] (N1) -- (N3);
            \draw [] (N1) -- (N4);
            \draw [] (N2) -- (N4);
            \draw [ultra thick] (N3) -- (N5);
            \draw [ultra thick] (N3) -- (N6);
            \draw [] (N4) -- (N6);
            \draw [ultra thick] (N5) -- (N6);
            \draw [] (N2) to [in=45, out=315] (N6);
            \draw [] (N1) -- (N7);
            \draw [] (N5) -- (N9);
            \draw [] (N7) -- (N8);
            \draw [] (N8) -- (N9);
        \end{scope}
        \begin{scope}[xshift=20cm]
            \node[draw, circle, fill=chromaygb5, inner sep=0.5pt] (Na) at (1,  0) {\phantom{0}};
            \node[draw, circle, fill=chromaygb5, inner sep=0.5pt] (Nb) at (1, -2) {\phantom{0}};
            \node[draw, circle, fill=chromaygb5, inner sep=0.5pt] (Nc) at (0, -4) {\phantom{0}};
            \node[draw, circle, fill=chromaygb5, inner sep=0.5pt] (Nd) at (2, -4) {\phantom{0}};

            \draw [ultra thick] (Na) -- (Nb);
            \draw [ultra thick] (Nb) -- (Nc);
            \draw [ultra thick] (Nc) -- (Nd);
            \draw [ultra thick] (Nb) -- (Nd);

            \node[draw, circle, fill=chromaygb5, inner sep=0.5pt] (N1) at (7.5,  0) {\phantom{0}};
            \node[draw, circle, fill=chromaygb5, inner sep=0.5pt] (N2) at (9.5,  0) {\phantom{0}};
            \node[draw, circle, fill=white, inner sep=0.5pt] (N3) at (7.5, -2) {\phantom{0}};
            \node[draw, circle, fill=chromaygb5, inner sep=0.5pt] (N4) at (9.5, -2) {\phantom{0}};
            \node[draw, circle, fill=white, inner sep=0.5pt] (N5) at (7.5, -4) {\phantom{0}};
            \node[draw, circle, fill=chromaygb5, inner sep=0.5pt] (N6) at (9.5, -4) {\phantom{0}};
            \node[draw, circle, fill=white, inner sep=0.5pt] (N7) at (5.5,  0) {\phantom{0}};
            \node[draw, circle, fill=white, inner sep=0.5pt] (N8) at (5.5, -2) {\phantom{0}};
            \node[draw, circle, fill=white, inner sep=0.5pt] (N9) at (5.5, -4) {\phantom{0}};

            \draw [ultra thick] (N1) -- (N2);
            \draw [] (N1) -- (N3);
            \draw [ultra thick] (N1) -- (N4);
            \draw [ultra thick] (N2) -- (N4);
            \draw [] (N3) -- (N5);
            \draw [] (N3) -- (N6);
            \draw [ultra thick] (N4) -- (N6);
            \draw [] (N5) -- (N6);
            \draw [ultra thick, dashed] (N2) to [in=45, out=315] (N6);
            \draw [] (N1) -- (N7);
            \draw [] (N5) -- (N9);
            \draw [] (N7) -- (N8);
            \draw [] (N8) -- (N9);
        \end{scope}
    \end{tikzpicture}

    \caption{On the left, an induced subgraph isomorphism. On the right, a non-induced subgraph
    isomorphism: the extra dashed edge is not present in the pattern graph.}
    \label{figure:sip}
\end{figure}

However, subgraph isomorphism algorithms cannot handle \emph{arbitrary} instances involving hundreds
or thousands of vertices. Experimental evaluations of subgraph isomorphism algorithms are usually
performed using randomly generated graph pairs, sometimes together with a mix of real-world graphs,
for example biochemistry and computer vision problems. Using random instances to evaluate
algorithm behaviour can be beneficial, because it provides a way of generating many instances
cheaply, and reduces the risk of over-fitting when tuning design parameters. It also allows
experimenters to control graph parameters such as order, density, or degree distribution, and to
study scaling properties of algorithms with respect to these parameters.  The random instances used
in previous evaluations come from common datasets
\citep{DBLP:journals/prl/SantoFSV03,DBLP:journals/constraints/ZampelliDS10}, which were generated by
taking a random subgraph of a random (Erd\H{o}s-R\'enyi, scale-free, bounded degree, or mesh) graph
and permuting the vertices. Such instances are guaranteed to be satisfiable---\citet{o:AntonO09}
exploited this property to create large sets of random satisfiable boolean satisfiability instances.
This is the most common approach to generating random subgraph isomorphism instances, meaning
existing benchmark suites contain relatively few non-trivial unsatisfiable instances (although a few
of the patterns in the instances by \citeauthor{DBLP:journals/constraints/ZampelliDS10} have had
extra edges added, making them unsatisfiable). Also, the satisfiable instances tend to be
computationally fairly easy, with most of the difficulty being in dealing with the size of the
model. This has lead to bias in algorithm design, to the extent that some proposed techniques, such
as those of \citet{DBLP:conf/sls/BattitiM07}, will \emph{only} work on satisfiable instances.

The first contribution of this paper is to present and evaluate new methods for creating random
pattern~/ target pairs.  The method we introduce in \cref{section:non-induced} generates both
satisfiable and unsatisfiable instances, and can produce computationally challenging instances with
only a few tens of vertices in the pattern, and 150 vertices in the target. Note that the lack of
unsatisfiable instances for testing purposes cannot be addressed simply by taking a pattern graph
from one of the existing random suites with the ``wrong'' target graph, as this tends to give either
a trivially unsatisfiable instance, or a satisfiable instance. (In particular, it is \emph{not} the
case that a relatively small random graph is unlikely to appear in a larger random graph.)

This work builds upon the phase transition phenomena observed in satisfiability and graph colouring
problems first described by \citet{DBLP:conf/ijcai/CheesemanKT91} and
\citet{DBLP:conf/aaai/MitchellSL92}. For subgraph isomorphism in Erd\H{o}s-R\'enyi random graphs, we
identify three relevant control parameters: we can independently alter the edge probability of the
pattern graph, the edge probability of the target graph, and the relative orders (number of
vertices) of the pattern and target graphs.  For non-induced isomorphisms, with the correct choice
of parameters we see results very similar to those observed with boolean satisfiability problems:
there is a phase transition (whose location we can predict) from satisfiable to unsatisfiable, and
we see a solver-independent complexity peak near this phase transition. Additionally, understanding
this behaviour helps us to design better variable- and value-ordering heuristics---this is the
second contribution of this paper.

In \cref{section:induced} we look at the induced isomorphism problem.  For certain choices of
parameters, there are two phase transitions, going from satisfiable to unsatisfiable, and then from
unsatisfiable back to satisfiable. Again, when going from satisfiable to unsatisfiable (from either
direction), instances go from being trivial to really hard to solve. However, each of the three
solvers we test also finds the central unsatisfiable region to be hard, despite it not being near a
phase transition. To show that this is not a simple weakness of current subgraph isomorphism
algorithms, we verify that this region is also hard for boolean satisfiability, pseudo-boolean, and
mixed integer solvers, and under reduction to the clique problem. Interestingly, the constrainedness
measure proposed by \citet{DBLP:conf/aaai/GentMPW96} \emph{does} predict this difficult region---the
third contribution of this paper is to use these instances to provide evidence in favour of
constrainedness, rather than proximity to a phase transition, being an accurate predictor of
difficulty, and to show that constrainedness is not simply a refinement of a phase transition
prediction.

What about other random graph models? In \cref{section:regular} we see that graphs where every vertex
has the same degree (known as $k$-regular) exhibit very similar characteristics.  However, when
labels on vertices are introduced (as is commonly seen in real-world applications and in graph
database systems), richer behaviour emerges, particularly when moving away from a uniform labelling
scheme, and the popular VF2 algorithm behaves much worse than two more recent
constraint-based algorithms in certain cases. In \cref{section:labelled} we take a close look at
these cases, and argue that they \emph{should} be easy to solve.

This is not simply a theoretical curiosity. A labelled version of subgraph isomorphism is one of
the key components in supporting complex queries in graph databases---typically, these systems store
a fixed set of target graphs, and for a sequence of pattern queries, they must return every target
graph which contains that pattern.  One common approach to this problem is to combine a subgraph
isomorphism algorithm with an indexing system in a so-called ``filter~/ verify'' model, where
invariants are used to attempt to pre-exclude unsatisfiable instances to avoid the cost of a
subgraph isomorphism call. In this context, the terms \emph{matching} and \emph{verification} are
often used for the subgraph isomorphism step (or a slightly broader problem, for example permitting
wildcards).

In \cref{section:filterverify} we look at some of the datasets commonly used to test graph database
systems, which the literature suggests are hard to solve. Simple experiments show that the entire
perceived difficulty of each of these datasets comes down to the widespread use of especially poor
subgraph isomorphism algorithms. Our fourth contribution is to show that the simplest constraint
programming approach, even without the more sophisticated recent advances in subgraph algorithms,
makes the entire graph database filter~/ verify paradigm unnecessary.  Finally, we explain why
filter~/ verify and other recently proposed techniques cannot be beneficial even on more challenging
instances except if an extremely poor choice of subgraph isomorphism algorithm is made.

\bigskip

Parts of this paper were presented at IJCAI \citep{DBLP:conf/ijcai/McCreeshPT16}; the first three
sections have been expanded, whilst
\cref{section:regular,section:labelled,section:filterverify} are new material.

\subsection{Definitions}

Throughout, our graphs are undirected, and do not have any loops. In some cases vertices have
labels, which are treated as integers; we write $\ell(v)$ for the label of vertex $v$. Two vertices
are \emph{adjacent} if there is an edge between them. The \emph{order} of a graph is the cardinality
of its vertex set. We write $\operatorname{V}(G)$ for the vertex set of a graph $G$. The
\emph{complement} of a graph $G$, denoted $\shortoverline{G}$, is the graph with the same vertex set
as $G$, and with an edge between distinct vertices $v$ and $w$ if and only if $v$ and $w$ are not
adjacent in $G$. We write $G(n, p)$ for an Erd\H{o}s-R\'enyi random graph with $n$ vertices, and an
edge between each distinct unordered pair of vertices with independent probability $p$.

A \emph{non-induced subgraph isomorphism} from a graph $P$ (called the \emph{pattern}) to a graph
$T$ (called the \emph{target}) is an injective mapping $i$ from $\operatorname{V}(P)$ to
$\operatorname{V}(T)$ which preserves adjacency---that is, for every adjacent vertices $v$ and $w$ in
$\operatorname{V}(P)$, the vertices $i(v)$ and $i(w)$ are adjacent in $T$. Where labels are present,
the mapping must also preserve labels, so $\ell(v) = \ell(i(v))$ for any $v$. An \emph{induced
subgraph isomorphism} additionally preserves non-adjacency---that is, if $v$ and $w$ are not
adjacent in $P$, then $i(v)$ and $i(w)$ must not be adjacent in $T$. We use the notation $i : P
\rightarrowtail T$ for a non-induced isomorphism, and $i : P \hookrightarrow T$ for an induced
isomorphism. Observe that an induced isomorphism $i : P \hookrightarrow T$ is a non-induced
isomorphism $i : P \rightarrowtail T$ which is also a non-induced isomorphism $\shortoverline{i} :
\shortoverline{P} \rightarrowtail \shortoverline{T}$.

\subsection{Experimental Setup}

The experiments in this paper are performed on systems with Intel Xeon E5-4650 v2 CPUs and 768GBytes
RAM, running Scientific Linux release 6.7. We will be working with four subgraph isomorphism
solvers: the Glasgow solver \citep{DBLP:conf/cp/McCreeshP15,DBLP:conf/lion/KotthoffMS16}; LAD
\citep{DBLP:journals/ai/Solnon10}; VF2 \citep{DBLP:journals/pami/CordellaFSV04}; and VF3
\citep{DBLP:conf/gbrpr/CarlettiFSV17}. Each was compiled using GCC 4.9 with the ``-O3'' option.

The Glasgow and LAD solvers are constraint programming inspired algorithms (although they use dedicated
data structures, and do not use a constraint programming toolkit). For each vertex in the pattern
graph, they create a \emph{variable} whose \emph{domain} contains one \emph{value} for each vertex
in the target graph. The algorithms perform backtracking search, with the goal of assigning to each
variable a value from its domain so as to form a subgraph isomorphism. The algorithms use
\emph{inference} to process \emph{constraints} which eliminate values from domains, ensuring that
the injectivity (an \emph{all-different} constraint) and adjacency rules are respected. If at any
point during search a \emph{domain wipeout} occurs (that is, if every value is removed from any
domain), the algorithms will backtrack immediately. The search is guided by \emph{variable-ordering
heuristics}, which select which variable (i.e.\ pattern vertex) to branch on next, and
\emph{value-ordering} heuristics which select which value (i.e.\ target vertex) to try as part of a
partial mapping. The two algorithms differ in terms of additional inference performed, their
choices of underlying data structures and algorithms, and in the details of their variable- and
value-ordering heuristics.

The approach used by VF2 and VF3 is a little different: domains of variables are not stored (so the
algorithm operates in the style of conventional backtracking), and wipeouts are not detected until
an assignment is made. Instead, an assignment is built up by repeatedly branching only on vertices
which are adjacent to a previously-assigned vertex.

We measure only the number of recursive calls (guessed assignments) made, not runtimes. We are not
aiming to compare absolute performance between solvers; rather, we are looking for
solver-independent patterns of difficulty. All experiments use a timeout of 1,000 seconds, which is
enough for the Glasgow solver to solve nearly all instances (whose orders were selected with this
timeout in mind), although we may slightly overestimate the proportion of unsatisfiable instances
for extremely sparse or dense pattern graphs. The LAD, VF2, and VF3 solvers experienced many more
failures with this timeout, so our picture of just how hard the hardest instances are with these
solvers is less detailed.

The Glasgow, LAD, and VF2 solvers support both non-induced and induced isomorphisms (although
neither Glasgow nor LAD were designed with the induced version in mind), whilst VF3 supports only
the induced problem.

\section{Non-Induced Subgraph Isomorphisms}\label{section:non-induced}

??

\subsection{A Phase Transitions when Varying Pattern Edge Density}

\begin{figure}[t]
    \centering
    \includegraphics*{plots/phase-transition.pdf}

    \caption{With a fixed pattern graph order of 20, a target graph order of 150, a target edge
    probability of 0.40, and varying pattern edge probability, we observe a phase transition and
    complexity peak with the Glasgow solver in the non-induced variant. Each point $(x, y)$
    represents one instance $i$, where $x$ is the pattern edge probability used to generate $i$, $y$
    is the number of search nodes needed by the Glasgow solver to solve $i$, and the point is drawn
    as a green circle if $i$ is satisfiable, and a blue cross otherwise. The black line plots the
    evolution of the mean number of search nodes when increasing the pattern edge probability from
    $0$ to $1$.}
    \label{figure:phase-transition}
\end{figure}

Suppose we arbitrarily decide upon a pattern graph order of 20, a target graph order of 150, a fixed
target edge probability of 0.40, and no vertex labels. As we vary the pattern edge probability from
0 to 1, we would expect to see a shift from entirely satisfiable instances (with no edges in the
pattern, we can always find a match) to entirely unsatisfiable instances (a maximum clique in this
order and edge probability of target graph will usually have between 9 and 12 vertices). The move
from green circles (satisfiable) to blue crosses (unsatisfiable) in \cref{figure:phase-transition}
shows that this is the case. For densities of 0.67 or greater, no instance is satisfiable; with
densities of 0.44 or less, every instance is satisfiable; and with a density of 0.55, roughly half
the instances are satisfiable.

The line plots mean search effort using the Glasgow solver: for sparse patterns, the problem is
trivial, for dense patterns proving unsatisfiability is not particularly difficult, and we see a
complexity peak around the point where half the instances are satisfiable.  We also plot the search
cost of individual instances, as points. The behaviour we observe looks remarkably similar to random
3SAT problems---compare, for example, Figure~1 of \citet{DBLP:journals/cacm/Leyton-BrownHHX14}. In
particular, satisfiable instances tend to be easier, but show greater variation than unsatisfiable
instances, and there are exceptionally hard satisfiable instances \citep{DBLP:conf/cp/SmithG97}.

\subsection{Phase Transitions when Varying Pattern and Target Edge Densities}

\begin{figure}[tb]
    \centering
    \begin{tikzpicture}[every node/.style={inner sep=0pt, outer sep=0pt}]
        \matrix {
            \node [anchor=center] {}; &
            \node [anchor=center] {\scriptsize $G(10, x) \rightarrowtail G(150, y)$\hspace*{1em}}; &
            \node [anchor=center] {\scriptsize $G(20, x) \rightarrowtail G(150, y)$\hspace*{1em}}; &
            \node [anchor=center] {\scriptsize $G(30, x) \rightarrowtail G(150, y)$\hspace*{3em}}; &
            \\
            \node [anchor=center, inner sep=2pt] [rotate=90] {\scriptsize Satisfiable?}; &
            \node [anchor=center] {\includegraphics*{plots/graph-non-induced-satisfiable-10-150.pdf}}; &
            \node [anchor=center] {\includegraphics*{plots/graph-non-induced-satisfiable-20-150.pdf}}; &
            \node [anchor=center] {\includegraphics*{plots/graph-non-induced-satisfiable-30-150.pdf}}; &
            \\[0.3cm]
            \node [anchor=center, inner sep=2pt] [rotate=90] {\scriptsize Glasgow}; &
            \node [anchor=center] {\includegraphics*{plots/graph-non-induced-nodes-10-150.pdf}}; &
            \node [anchor=center] {\includegraphics*{plots/graph-non-induced-nodes-20-150.pdf}}; &
            \node [anchor=center] {\includegraphics*{plots/graph-non-induced-nodes-30-150.pdf}}; &
            \\[0.3cm]
            \node [anchor=center, inner sep=2pt] [rotate=90] {\scriptsize LAD}; &
            \node [anchor=center] {\includegraphics*{plots/graph-lad-non-induced-nodes-10-150.pdf}}; &
            \node [anchor=center] {\includegraphics*{plots/graph-lad-non-induced-nodes-20-150.pdf}}; &
            \node [anchor=center] {\includegraphics*{plots/graph-lad-non-induced-nodes-30-150.pdf}}; &
            \\[0.3cm]
            \node [anchor=center, inner sep=2pt] [rotate=90] {\scriptsize VF2}; &
            \node [anchor=center] {\includegraphics*{plots/graph-vf2-non-induced-nodes-10-150.pdf}}; &
            \node [anchor=center] {\includegraphics*{plots/graph-vf2-non-induced-nodes-20-150.pdf}}; &
            \node [anchor=center] {\includegraphics*{plots/graph-vf2-non-induced-nodes-30-150.pdf}}; \\
        };
    \end{tikzpicture}

    \caption{Behaviour of algorithms on the non-induced variant, using target graphs of 150
        vertices, and pattern graphs of 10, 20, or 30 vertices. For each plot, the x-axis is the
        pattern edge probability and the y-axis is the target edge probability, both from 0 to 1.
        Along the top row, we show the proportion of instances which are satisfiable; the white
        bands show the phase transitions, and the black lines are the predictions using
        equation~\eqref{equation:non-induced-prediction} of where the phase transition will occur.
        On the subsequent three rows, we show the average number of search nodes used by the
        Glasgow, LAD and VF2 solvers over ten instances; the dark regions indicate ``really hard''
        instances, and black
        points indicate that at least one timeout occurred.}
    \label{figure:non-induced}
\end{figure}

What if we alter the edge probabilities for both the pattern graph and the target graph?  In the top
row of \cref{figure:non-induced} we show the satisfiability phase transition for the non-induced
variant, for patterns of order 10, 20 and 30, targets of order 150, and varying pattern (x-axis) and
target (y-axis) edge probabilities. Each axis runs over 101 edge probabilities, from 0 to 1 in steps
of 0.01. For each of these points, we generate ten random instances. The colour denotes the
proportion of these instances which were found to be satisfiable.  Inside the red region, at the
bottom right of each plot, every instance is unsatisfiable---here we are trying to find a dense
pattern in a sparse target. In the green region, at the top left, every instance is
satisfiable---we are looking for a sparse pattern in a dense target (which is easy, since we only
have to preserve adjacency, not non-adjacency). The white band between the regions shows the
location of the phase transition: here, roughly half the instances are satisfiable. (We discuss the
black line below.)

On subsequent rows, we show the average number of search nodes used by the different algorithms.
Darker regions indicate harder difficulties, and black is used to indicate that at least one timeout
occurred. In general, satisfiable instances are easy, until very close to the phase transition. As
we hit the phase transition and move into the unsatisfiable region, we see complexity increase.
Finally, as we pass through the phase transition and move deeper into the unsatisfiable region,
instances become easier again. This behaviour is largely solver-independent, although VF2 has a
larger hard region than Glasgow or LAD. Thus, although we have moved away from a single control
parameter, we still observe the easy-hard-easy pattern seen in many \NP-complete problems.

Finally, a close look at the very bottom left of the plots shows that VF2 will sometimes give a
timeout on instances where the target is empty but the pattern is not---this turns out to be
extremely important, and we return to it in \cref{section:labelled,section:filterverify}.

\subsection{Locating the Phase Transition}

We can approximately predict the location of the phase transition by calculating (with
simplifications regarding rounding and independence) the expected number of solutions for given
parameters. Since we are trying to find an \emph{injective} mapping from a pattern $P = G(p, d_p)$
to a target $T = G(t, d_t)$, there are \[ t^{\underline{p}} = t \cdot (t - 1) \cdot \ldots \cdot (t
- p + 1) \] possible injective assignments of target vertices to pattern vertices.  If we assume the
pattern has exactly $d_p \cdot \binom{p}{2}$ edges, we obtain the probability of all of these edges
being mapped to edges in the target by raising $d_t$ to this power, giving an expected number of
solutions of \begin{equation}\label{equation:non-induced-prediction} \langle Sol \rangle =
t^{\underline{p}} \cdot {d_t}^{d_p \cdot \binom{p}{2}} \textnormal{.} \end{equation} This formula
predicts a very sharp phase transition from $\langle Sol \rangle \ll 1$ to $\langle Sol \rangle \gg
1$, which may easily be located numerically. We plot where this occurs using black lines in the
first row of \cref{figure:non-induced}.

This prediction is generally reasonably accurate, except that for very low and very high pattern
densities, we overestimate the satisfiable region. This is due to variance: although an expected
number of solutions much below one implies a high likelihood of unsatisfiability, it is not true
that a high expected number of solutions implies that any particular instance is likely to be
satisfiable. (Consider, for example, a sparse graph which has several isolated vertices. If one
solution exists, other symmetric solutions can be obtained by permuting the isolated vertices.
Thus although the expected number of solutions may be one, there cannot be exactly one solution.) A
similar behaviour is seen with random constraint satisfaction problems
\citep{DBLP:journals/ai/SmithD96}.

\subsection{Variable- and Value-Ordering Heuristics}

Various general principles have been considered when designing variable- and value-ordering
heuristics for backtracking search algorithms---one of these is to try to maximise the expected
number of solutions inside any subproblem considered during search \citep{DBLP:conf/cp/GentMPSW96}.
This is usually done by cheaper surrogates, rather than direct calculation. When branching, both LAD
and Glasgow pick a variable with fewest remaining values in its domain: doing this will generally
reduce the first part of the $\langle Sol \rangle$ equation (i.e.\ $t^p$) by as little as possible.
When two or more domains are of equal size, LAD breaks ties lexicographically, whereas Glasgow will
pick a variable corresponding to a pattern vertex of highest degree. This strategy was determined
empirically, but could have been derived from the $\langle Sol \rangle$ formula: picking a pattern
vertex of high degree will make the remaining pattern subgraph sparser, which will decrease the
exponent in the second half of the formula, maximising the overall value. LAD does not apply a
value-ordering heuristic, but Glasgow does: it prefers target vertices of highest degree.  Again,
this was determined empirically, but it has the effect of increasing $\langle Sol \rangle$ by
leaving as many vertices as possible available for future use. The search strategy used by VF2, in
contrast, is based around preserving connectivity, which gives very little discrimination except on
the sparsest of inputs.

\section{Induced Subgraph Isomorphisms}\label{section:induced}

\begin{figure}[p]
    \centering
    \begin{tikzpicture}[every node/.style={inner sep=0pt, outer sep=0pt}]
        \matrix {
            \node [anchor=center] {}; &
            \node [anchor=center] {\scriptsize $G(10, x) \hookrightarrow$\hspace*{1em}}; &
            \node [anchor=center] {\scriptsize $G(14, x) \hookrightarrow$\hspace*{1em}}; &
            \node [anchor=center] {\scriptsize $G(15, x) \hookrightarrow$\hspace*{1em}}; &
            \node [anchor=center] {\scriptsize $G(16, x) \hookrightarrow$\hspace*{1em}}; &
            \node [anchor=center] {\scriptsize $G(20, x) \hookrightarrow$\hspace*{1em}}; &
            \node [anchor=center] {\scriptsize $G(30, x) \hookrightarrow$\hspace*{3em}}; &
            \\
            \node [anchor=center] {}; &
            \node [anchor=center] {\scriptsize $G(150, y)$\hspace*{1em}}; &
            \node [anchor=center] {\scriptsize $G(150, y)$\hspace*{1em}}; &
            \node [anchor=center] {\scriptsize $G(150, y)$\hspace*{1em}}; &
            \node [anchor=center] {\scriptsize $G(150, y)$\hspace*{1em}}; &
            \node [anchor=center] {\scriptsize $G(150, y)$\hspace*{1em}}; &
            \node [anchor=center] {\scriptsize $G(150, y)$\hspace*{3em}}; &
            \\[-0.08cm]
            \node [anchor=center, inner sep=2pt, rotate=90] {\scriptsize Satisfiable?}; &
            \node [anchor=center] {\includegraphics*{plots/graph-induced-satisfiable-10-150.pdf}}; &
            \node [anchor=center] {\includegraphics*{plots/graph-induced-satisfiable-14-150.pdf}}; &
            \node [anchor=center] {\includegraphics*{plots/graph-induced-satisfiable-15-150.pdf}}; &
            \node [anchor=center] {\includegraphics*{plots/graph-induced-satisfiable-16-150.pdf}}; &
            \node [anchor=center] {\includegraphics*{plots/graph-induced-satisfiable-20-150.pdf}}; &
            \node [anchor=center] {\includegraphics*{plots/graph-induced-satisfiable-30-150.pdf}}; &
            \\[-0.08cm]
            \node [anchor=center, inner sep=2pt, rotate=90] {\scriptsize Glasgow}; &
            \node [anchor=center] {\includegraphics*{plots/graph-induced-nodes-10-150.pdf}}; &
            \node [anchor=center] {\includegraphics*{plots/graph-induced-nodes-14-150.pdf}}; &
            \node [anchor=center] {\includegraphics*{plots/graph-induced-nodes-15-150.pdf}}; &
            \node [anchor=center] {\includegraphics*{plots/graph-induced-nodes-16-150.pdf}}; &
            \node [anchor=center] {\includegraphics*{plots/graph-induced-nodes-20-150.pdf}}; &
            \node [anchor=center] {\includegraphics*{plots/graph-induced-nodes-30-150.pdf}}; &
            \\[-0.08cm]
            \node [anchor=center, inner sep=2pt, rotate=90] {\scriptsize LAD}; &
            \node [anchor=center] {\includegraphics*{plots/graph-lad-induced-nodes-10-150.pdf}}; &
            \node [anchor=center] {\includegraphics*{plots/graph-lad-induced-nodes-14-150.pdf}}; &
            \node [anchor=center] {\includegraphics*{plots/graph-lad-induced-nodes-15-150.pdf}}; &
            \node [anchor=center] {\includegraphics*{plots/graph-lad-induced-nodes-16-150.pdf}}; &
            \node [anchor=center] {\includegraphics*{plots/graph-lad-induced-nodes-20-150.pdf}}; &
            \node [anchor=center] {\includegraphics*{plots/graph-lad-induced-nodes-30-150.pdf}}; &
            \\[-0.08cm]
            \node [anchor=center, inner sep=2pt, rotate=90] {\scriptsize VF2}; &
            \node [anchor=center] {\includegraphics*{plots/graph-vf2-induced-nodes-10-150.pdf}}; &
            \node [anchor=center] {\includegraphics*{plots/graph-vf2-induced-nodes-14-150.pdf}}; &
            \node [anchor=center] {\includegraphics*{plots/graph-vf2-induced-nodes-15-150.pdf}}; &
            \node [anchor=center] {\includegraphics*{plots/graph-vf2-induced-nodes-16-150.pdf}}; &
            \node [anchor=center] {\includegraphics*{plots/graph-vf2-induced-nodes-20-150.pdf}}; &
            \node [anchor=center] {\includegraphics*{plots/graph-vf2-induced-nodes-30-150.pdf}}; \\
            \\[-0.08cm]
            \node [anchor=center, inner sep=2pt, rotate=90] {\scriptsize VF3}; &
            \node [anchor=center] {\includegraphics*{plots/graph-vf3-induced-nodes-10-150.pdf}}; &
            \node [anchor=center] {\includegraphics*{plots/graph-vf3-induced-nodes-14-150.pdf}}; &
            \node [anchor=center] {\includegraphics*{plots/graph-vf3-induced-nodes-15-150.pdf}}; &
            \node [anchor=center] {\includegraphics*{plots/graph-vf3-induced-nodes-16-150.pdf}}; &
            \node [anchor=center] {\includegraphics*{plots/graph-vf3-induced-nodes-20-150.pdf}}; &
            \node [anchor=center] {\includegraphics*{plots/graph-vf3-induced-nodes-30-150.pdf}}; \\
            \\[-0.08cm]
            \node [anchor=center, inner sep=2pt, rotate=90] {\scriptsize Constrainedness}; &
            \node [anchor=center] {\includegraphics*{plots/graph-induced-kappa-10-150.pdf}}; &
            \node [anchor=center] {\includegraphics*{plots/graph-induced-kappa-14-150.pdf}}; &
            \node [anchor=center] {\includegraphics*{plots/graph-induced-kappa-15-150.pdf}}; &
            \node [anchor=center] {\includegraphics*{plots/graph-induced-kappa-16-150.pdf}}; &
            \node [anchor=center] {\includegraphics*{plots/graph-induced-kappa-20-150.pdf}}; &
            \node [anchor=center] {\includegraphics*{plots/graph-induced-kappa-30-150.pdf}}; \\
            \\[-0.08cm]
            \node [anchor=center, inner sep=2pt, rotate=90] {\scriptsize Complement?}; &
            \node [anchor=center] {\includegraphics*{plots/graph-induced-which-10-150.pdf}}; &
            \node [anchor=center] {\includegraphics*{plots/graph-induced-which-14-150.pdf}}; &
            \node [anchor=center] {\includegraphics*{plots/graph-induced-which-15-150.pdf}}; &
            \node [anchor=center] {\includegraphics*{plots/graph-induced-which-16-150.pdf}}; &
            \node [anchor=center] {\includegraphics*{plots/graph-induced-which-20-150.pdf}}; &
            \node [anchor=center] {\includegraphics*{plots/graph-induced-which-30-150.pdf}}; \\
        };
    \end{tikzpicture}
    \caption{Behaviour of algorithms on the induced variant with target graphs of 150 vertices,
    shown in the style of \cref{figure:non-induced}. The sixth row plots constrainedness using
    equation~\eqref{equation:constrainedness}: the darkest region is where $\kappa = 1$, and the
    lighter regions show where the problem is either over- or under-constrained. The final row shows
    when the Glasgow algorithm performs better when given the complements of the pattern and target
    graphs as inputs---the solid lines show the empirical location of the phase transition, and the
    dotted lines are $t_d=0.5$ and the $p_d=t_d$ diagonal.}\label{figure:induced}
\end{figure}

So far we have looked at the non-induced problem, where extra edges in the target graph are
permitted.  In the first five rows of \cref{figure:induced} we repeat our experiments, now finding
induced isomorphisms. With a pattern of order 10, we get two independent phase transitions: the
bottom right halves of the plots resemble the non-induced results, and the top left halves are close
to a mirror image. This second phase transition comes from the fact that an empty pattern is not an
induced subgraph of a complete target: an empty pattern of order $k$ is an induced subgraph only if
there exist $k$ vertices in the target that are not pairwise connected.  The central satisfiable
region, which is away from either phase transition, is computationally easy, but instances near the
phase transition are hard.

For larger patterns of order 20 and 30, we have a large unsatisfiable region in the middle. Despite
not being near either phase transition, instances in the centre remain computationally challenging
for every solver.  We also plot patterns of orders 14, 15 and 16, to show the change between the two
behaviours.

We might expect these complexity plots to be symmetric along the diagonal, since for the induced
problem, if we replace both inputs with their complements, the solutions remain the same.  For the
Glasgow solver, this is almost the case. This should be expected, because this complement property is
precisely how the Glasgow solver handles the induced variant (although the heuristics may differ
between the two, which we discuss below). For LAD, some of the very dense patterns are slightly
harder than their diagonal opposites (LAD reasons about degrees, but not about complement-degrees).

It is interesting to note that for larger target graphs, VF2 finds \emph{all} dense pattern graphs
difficult. Meanwhile, VF3 behaves better than VF2 in some regions, but worse than VF2 in others, and
has a larger hard region than either domain-based algorithm.\footnote{The version of
VF3 supplied by its inventors will crash if given an empty pattern graph. We have chosen to treat
these results as successes taking zero search, even though empty pattern graphs are not trivial for
the induced problem. If the reader prefers to interpret a crash as a timeout, the VF3 plots would
contain an additional area of black points down the left-hand side.} VF3 also occasionally finds
some instances with sparse target graphs (along the bottom of the plot) extremely difficult, whilst
other solvers do not.

\subsection{Predictions and Heuristics}

To predict the location of the induced phase transition, we repeat the argument for locating the
non-induced phase transition and additionally consider non-edges, to get an expected number of
solutions of \begin{equation}\label{equation:induced-prediction} \langle Sol \rangle = t^{\underline{p}} \cdot {d_t}^{d_p \cdot \binom{p}{2}} \cdot
{(1 - d_{t})}^{(1 - d_{p}) \cdot \binom{p}{2}} \textnormal{.} \end{equation} We plot this using black lines on
the top row of \cref{figure:induced}---again, our prediction is accurate except for very sparse or
very dense patterns.

We might guess that degree-based heuristics would just not work for the induced problem: for any
claim about the degree, the opposite will hold for the complement constraints. Should we therefore
resort to just using ``smallest domain first'', abandoning degree-based tiebreaking and the
value-ordering heuristic? Empirically, this is not the case: on the final row of
\cref{figure:induced}, we show whether it is better to use the original pattern and target as the
input to the Glasgow algorithm, or to take the complements.  (The only steps performed by the
Glagsow algorithm which differ under taking the complements are the degree-based heuristics.  LAD,
VF2, and VF3 are not symmetric in this way: LAD performs a filtering step using degree information,
but does not consider the complement degree, and VF2 and VF3 use connectivity.)

For patterns of order 10, it is always better to try to move towards the satisfiable region: if we
are in the bottom right diagonal half, we are best retaining the original heuristics (which move us
towards the top left), and if we are in the top left we should use the complement instead. This
goes against a suggestion by \citet{DBLP:conf/aaai/Walsh98} that switching heuristics based upon an estimate of
the solubility of the problem may offer good performance.

For larger patterns, more complex behaviour emerges. If we are in the intersection of the bottom half
and the bottom right diagonal of the search space, we should always retain the original heuristic,
and if we are in the intersection of the top half and the top left diagonal, we should always use
the complements. This behaviour can be predicted by taking the partial derivatives of $\langle Sol
\rangle$ in the $-p_d$ and $t_d$ directions.  However, when inside the remaining two eighths of the
parameter space, the partial derivatives of $\langle Sol \rangle$ disagree on which heuristic to
use, and using directional derivatives is not enough to resolve the problem. A close observation of
the data suggests that the actual location of the phase transition may be involved (and perhaps
\citeauthor{DBLP:conf/aaai/Walsh98}'s suggestion applies only in these conditions). In any case, $\langle Sol
\rangle$ is insufficient to explain the observed behaviour in these two eighths of the parameter space.

In practice, this is unlikely to be a problem: most of the real-world instances we have seen tend to
be relatively sparse. In this situation, these experiments justify reusing the non-induced
heuristics on induced problems.

\subsection{Is the Central Region Genuinely Hard?}

\begin{figure}[p]
    \centering
    \begin{tikzpicture}[every node/.style={inner sep=0pt, outer sep=0pt}]
        \matrix {
            \node [anchor=center] {}; &
            \node [anchor=center] {\scriptsize $G(10, x) \hookrightarrow$\hspace*{1em}}; &
            \node [anchor=center] {\scriptsize $G(12, x) \hookrightarrow$\hspace*{1em}}; &
            \node [anchor=center] {\scriptsize $G(14, x) \hookrightarrow$\hspace*{1em}}; &
            \node [anchor=center] {\scriptsize $G(16, x) \hookrightarrow$\hspace*{1em}}; &
            \node [anchor=center] {\scriptsize $G(18, x) \hookrightarrow$\hspace*{1em}}; &
            \node [anchor=center] {\scriptsize $G(25, x) \hookrightarrow$\hspace*{3em}}; &
            \\
            \node [anchor=center] {}; &
            \node [anchor=center] {\scriptsize $G(75, y)$\hspace*{1em}}; &
            \node [anchor=center] {\scriptsize $G(75, y)$\hspace*{1em}}; &
            \node [anchor=center] {\scriptsize $G(75, y)$\hspace*{1em}}; &
            \node [anchor=center] {\scriptsize $G(75, y)$\hspace*{1em}}; &
            \node [anchor=center] {\scriptsize $G(75, y)$\hspace*{1em}}; &
            \node [anchor=center] {\scriptsize $G(75, y)$\hspace*{3em}}; &
            \\
            \node [anchor=center, inner sep=2pt, rotate=90] {\scriptsize Satisfiable?}; &
            \node [anchor=center] {\includegraphics*{plots/graph-induced-satisfiable-10-75.pdf}}; &
            \node [anchor=center] {\includegraphics*{plots/graph-induced-satisfiable-12-75.pdf}}; &
            \node [anchor=center] {\includegraphics*{plots/graph-induced-satisfiable-14-75.pdf}}; &
            \node [anchor=center] {\includegraphics*{plots/graph-induced-satisfiable-16-75.pdf}}; &
            \node [anchor=center] {\includegraphics*{plots/graph-induced-satisfiable-18-75.pdf}}; &
            \node [anchor=center] {\includegraphics*{plots/graph-induced-satisfiable-25-75.pdf}}; &
            \\[0.3cm]
            \node [anchor=center, inner sep=2pt, rotate=90] {\scriptsize Glasgow}; &
            \node [anchor=center] {\includegraphics*{plots/graph-induced-nodes-10-75.pdf}}; &
            \node [anchor=center] {\includegraphics*{plots/graph-induced-nodes-12-75.pdf}}; &
            \node [anchor=center] {\includegraphics*{plots/graph-induced-nodes-14-75.pdf}}; &
            \node [anchor=center] {\includegraphics*{plots/graph-induced-nodes-16-75.pdf}}; &
            \node [anchor=center] {\includegraphics*{plots/graph-induced-nodes-18-75.pdf}}; &
            \node [anchor=center] {\includegraphics*{plots/graph-induced-nodes-25-75.pdf}}; &
            \\[0.3cm]
            \node [anchor=center, inner sep=2pt, rotate=90] {\scriptsize Clasp (PB)}; &
            \node [anchor=center] {\includegraphics*{plots/graph-clasp-induced-nodes-10-75.pdf}}; &
            \node [anchor=center] {\includegraphics*{plots/graph-clasp-induced-nodes-12-75.pdf}}; &
            \node [anchor=center] {\includegraphics*{plots/graph-clasp-induced-nodes-14-75.pdf}}; &
            \node [anchor=center] {\includegraphics*{plots/graph-clasp-induced-nodes-16-75.pdf}}; &
            \node [anchor=center] {\includegraphics*{plots/graph-clasp-induced-nodes-18-75.pdf}}; &
            \node [anchor=center] {\includegraphics*{plots/graph-clasp-induced-nodes-25-75.pdf}}; &
            \\[0.3cm]
            \node [anchor=center, inner sep=2pt, rotate=90] {\scriptsize Glucose (SAT)}; &
            \node [anchor=center] {\includegraphics*{plots/graph-glucose-induced-nodes-10-75.pdf}}; &
            \node [anchor=center] {\includegraphics*{plots/graph-glucose-induced-nodes-12-75.pdf}}; &
            \node [anchor=center] {\includegraphics*{plots/graph-glucose-induced-nodes-14-75.pdf}}; &
            \node [anchor=center] {\includegraphics*{plots/graph-glucose-induced-nodes-16-75.pdf}}; &
            \node [anchor=center] {\includegraphics*{plots/graph-glucose-induced-nodes-18-75.pdf}}; &
            \node [anchor=center] {\includegraphics*{plots/graph-glucose-induced-nodes-25-75.pdf}}; &
            \\[0.3cm]
            \node [anchor=center, inner sep=2pt, rotate=90] {\scriptsize Gurobi (MIP)}; &
            \node [anchor=center] {\includegraphics*{plots/graph-gurobi-induced-nodes-10-75.pdf}}; &
            \node [anchor=center] {\includegraphics*{plots/graph-gurobi-induced-nodes-12-75.pdf}}; &
            \node [anchor=center] {\includegraphics*{plots/graph-gurobi-induced-nodes-14-75.pdf}}; &
            \node [anchor=center] {\includegraphics*{plots/graph-gurobi-induced-nodes-16-75.pdf}}; &
            \node [anchor=center] {\includegraphics*{plots/graph-gurobi-induced-nodes-18-75.pdf}}; &
            \node [anchor=center] {\includegraphics*{plots/graph-gurobi-induced-nodes-25-75.pdf}}; &
            \\[0.3cm]
            \node [anchor=center, inner sep=2pt, rotate=90] {\scriptsize BBMC (Clique)}; &
            \node [anchor=center] {\includegraphics*{plots/graph-clique-induced-nodes-10-75.pdf}}; &
            \node [anchor=center] {\includegraphics*{plots/graph-clique-induced-nodes-12-75.pdf}}; &
            \node [anchor=center] {\includegraphics*{plots/graph-clique-induced-nodes-14-75.pdf}}; &
            \node [anchor=center] {\includegraphics*{plots/graph-clique-induced-nodes-16-75.pdf}}; &
            \node [anchor=center] {\includegraphics*{plots/graph-clique-induced-nodes-18-75.pdf}}; &
            \node [anchor=center] {\includegraphics*{plots/graph-clique-induced-nodes-25-75.pdf}}; &
            \\
        };
    \end{tikzpicture}
    \caption{Behaviour of other solvers on the induced variant using smaller target graphs with 75
        vertices, shown in the style of \cref{figure:non-induced}. The second row shows the number
        of search nodes used by the Glasgow algorithm, the third and fourth rows show the number of
        decisions made by the pseudo-boolean and SAT solvers, the fifth shows the number of search
        nodes used on the MIP encoding, and the
    final row the clique encoding.}\label{figure:alt}
\end{figure}

The region in the parameter space where both pattern and target have medium density is far from a
phase transition, but nevertheless contains instances that are hard for all four solvers. We would
like to know whether this is due to a weakness in current solvers (perhaps our solvers cannot reason
about adjacency and non-adjacency simultaneously?), or whether instances in this region are
inherently difficult to solve.  Thus we repeat the induced experiments on smaller pattern and target
graphs, using different solving techniques.  Although these techniques are not competitive in
absolute terms, we wish to see if the same pattern of behaviour occurs. The results are plotted in
\cref{figure:alt}.

The pseudo-boolean (PB) encoding is as follows. For each pattern vertex $v$ and each target vertex
$w$, we have a binary variable which takes the value 1 if and only if $v$ is mapped to $w$.
Constraints are added to ensure that each pattern vertex maps to exactly one target vertex, that
each target vertex is mapped to by at most one pattern vertex, that adjacent vertices are mapped to
adjacent vertices, and that non-adjacent vertices are mapped to non-adjacent vertices. We use the
Clasp solver \citep{DBLP:journals/aicom/GebserKKOSS11} version 3.1.3 to solve the pseudo-boolean instances.  The
instances that are hard for the Glasgow solver remain hard for the PB solver, including instances
inside the central region, and the easy satisfiable instances remain easy. Similar results are seen
with the Glucose SAT solver \citep{o:glucose} using a direct encoding.  We also show an integer
program encoding: the Gurobi solver is only able to solve some of the trivial satisfiable instances,
and was almost never able to prove unsatisfiability within the time limit.

The \emph{association graph encoding} of a subgraph isomorphism problem is a reduction to the clique
decision problem. \citet{DBLP:conf/cp/McCreeshNPS16} describe and study this approach in more
detail. Briefly, the association graph is constructed by creating a new graph with a vertex for each
pair $(p, t)$ of vertices from the pattern and target graphs respectively. There is an edge between
vertex $(p_1, t_1)$ and vertex $(p_2, t_2)$ if mapping $p_1$ to $t_1$ and $p_2$ to $t_2$
simultaneously is permitted, i.e.\ $p_1$ is adjacent to $p_2$ if and only if $t_1$ is adjacent to
$t_2$. A clique of size equal to the order of the pattern graph exists in the association graph if
and only if the problem is satisfiable \citep{o:Levi73}. We used this encoding with an
implementation of \citet{DBLP:journals/cor/SegundoRJ11}'s bit-parallel maximum clique algorithm
BBMC, modified to solve the decision problem rather than the optimisation problem.  Again, our
results show that the instances in the central region remain hard, and additionally, some of the
easy unsatisfiable instances become hard.

Together, these experiments suggest that the central region may be genuinely hard, despite not being
near a phase transition. The clique results in particular rule out the hypothesis that subgraph
isomorphism solvers only find this region hard due to not reasoning simultaneously about adjacency
and non-adjacency, since the constraints in the association graph encoding consider compatibility
rather than adjacency and non-adjacency.

\subsection{Constrainedness}

Constrainedness, denoted $\kappa$, is an alternative measure of difficulty designed to refine the
phase transition concept, and to generalise hardness parameters across different combinatorial
problems \citep{DBLP:conf/aaai/GentMPW96}. A problem with $\kappa < 1$ is said to be
underconstrained, and is likely to be satisfiable; a problem with $\kappa > 1$ is overconstrained,
and is likely to be unsatisfiable. Empirically, problems with $\kappa$ close to 1 are hard, and
problems where $\kappa$ is very small or very large are usually easy. By handling injectivity as a
restriction on the size of the state space rather than as a constraint, we derive
\begin{equation}\label{equation:constrainedness}\kappa = 1 - \frac{\log \left( t^{\underline{p}}
\cdot {d_t}^{d_p \cdot \binom{p}{2}} \cdot {(1 - d_{t})}^{(1 - d_{p}) \cdot \binom{p}{2}}
\right)}{\log t^{\underline{p}}} \end{equation} for induced isomorphisms, which we plot on the sixth
row of \cref{figure:induced}. We see that constrainedness predicts that the central region will
still be relatively difficult for larger pattern graphs: although the problem is overconstrained, it
is less overconstrained than in the regions the Glasgow and LAD solvers found easy.  Thus it seems
that rather than just being a unification of existing generalised heuristic techniques,
constrainedness also gives a better predictor of difficulty than proximity to a phase
transition---our method generates instances where constrainedness and ``close to a phase
transition'' give very different predictions, and constrainedness closely matches the empirical
results.

Unfortunately, constrainedness does not give additional heuristic information: minimising
constrainedness gives the same predictions as maximising the expected number of solutions.

\section{$k$-Regular Graphs}\label{section:regular}

\begin{figure}[tb]
    \centering
    \begin{tikzpicture}[every node/.style={inner sep=0pt, outer sep=0pt}]
        \matrix {
            \node [anchor=center] {}; &
            \node [anchor=center] {\scriptsize $R(10, x) \rightarrowtail R(150, y)$\hspace*{1em}}; &
            \node [anchor=center] {\scriptsize $R(20, x) \rightarrowtail R(150, y)$\hspace*{1em}}; &
            \node [anchor=center] {\scriptsize $R(30, x) \rightarrowtail R(150, y)$\hspace*{3em}}; &
            \\
            \node [anchor=center, inner sep=2pt] [rotate=90] {\scriptsize Satisfiable?}; &
            \node [anchor=center] {\includegraphics*{plots/graph-non-induced-kr-satisfiable-10-150.pdf}}; &
            \node [anchor=center] {\includegraphics*{plots/graph-non-induced-kr-satisfiable-20-150.pdf}}; &
            \node [anchor=center] {\includegraphics*{plots/graph-non-induced-kr-satisfiable-30-150.pdf}}; &
            \\[0.3cm]
            \node [anchor=center, inner sep=2pt] [rotate=90] {\scriptsize Glasgow}; &
            \node [anchor=center] {\includegraphics*{plots/graph-non-induced-kr-nodes-10-150.pdf}}; &
            \node [anchor=center] {\includegraphics*{plots/graph-non-induced-kr-nodes-20-150.pdf}}; &
            \node [anchor=center] {\includegraphics*{plots/graph-non-induced-kr-nodes-30-150.pdf}}; &
            \\[0.3cm]
            \node [anchor=center, inner sep=2pt] [rotate=90] {\scriptsize LAD}; &
            \node [anchor=center] {\includegraphics*{plots/graph-lad-non-induced-kr-nodes-10-150.pdf}}; &
            \node [anchor=center] {\includegraphics*{plots/graph-lad-non-induced-kr-nodes-20-150.pdf}}; &
            \node [anchor=center] {\includegraphics*{plots/graph-lad-non-induced-kr-nodes-30-150.pdf}}; &
            \\[0.3cm]
            \node [anchor=center, inner sep=2pt] [rotate=90] {\scriptsize VF2}; &
            \node [anchor=center] {\includegraphics*{plots/graph-vf2-non-induced-kr-nodes-10-150.pdf}}; &
            \node [anchor=center] {\includegraphics*{plots/graph-vf2-non-induced-kr-nodes-20-150.pdf}}; &
            \node [anchor=center] {\includegraphics*{plots/graph-vf2-non-induced-kr-nodes-30-150.pdf}}; &
            \\
        };
    \end{tikzpicture}

    \caption{Behaviour of algorithms on the non-induced problem on random $k$-regular graphs, as the
    pattern degree (x-axis) and target degree (y-axis) are varied from 0 to
    $\left|\operatorname{V}(G)\right| - 1$, using target graphs of 150 vertices, and pattern graphs
    of 10, 20, and 30 vertices.}
    \label{figure:kr}
\end{figure}

What about richer graph structures? In this section we briefly look at what happens with the
non-induced problem if we use $k$-regular graphs rather than the Erd\H{o}s-R\'enyi model, and then
in the following section we see the effects of introducing vertex labels. Both models still exhibit
phase transition behaviour, although with sufficiently many labels the ``satisfiable'' region now
contains a mix of satisfiable and unsatisfiable instances.

By $R(n, k)$ we mean a random graph with $n$ vertices, each of which has degree $k$. In
\cref{figure:kr} we recreate \cref{figure:non-induced}, using the NetworkX implementation
\citep{o:NetworkX} of the \citet{DBLP:journals/cpc/StegerW99} algorithm to generate regular graphs.
The $x$-axes range from degree 0 to degree $n - 1$ where $n \in \{ 10, 20, 30 \}$ is the number of
pattern vertices, and the y-axes range from degree 0 to degree
149 (one less than the number of target vertices).  The satisfiable / unsatisfiable plots show very
similar results to the non-induced problem on Erd\H{o}s-R\'enyi random graphs, with a similarly
sharp phase transition, and again only instances near the phase transition are difficult.

Regularity means that degree-based heuristics have no information to work with. Additionally, the
degree-based filtering techniques used by Glasgow and LAD have no effect on these graphs at the top
of search, and so initially every domain is identical. However, once a guessed assignment has been
made, selecting from small domains first remains an effective strategy to guide the remainder of the
search.

\section{Labelled Graphs}\label{section:labelled}

So far, we have looked at unlabelled graphs. What happens when labels on vertices are introduced?
This is common in real-world applications---for example, when working with graphs representing
chemical molecules, mappings are typically expected only to match carbon atoms with carbon atoms,
hydrogen atoms with hydrogen atoms, and so on.  We will look at the non-induced variant, as this
seems to be more common in the literature.
% , and return to the Erd\H{o}s-R\'enyi model.

\subsection{Predictions and Empirical Hardness}

Suppose our labels are drawn randomly from a set $L = \{ 1, \ldots, k \}$, where $k$ is reasonably
small compared to the number of pattern vertices $p$. Recall that $\ell(v)$ is the label of vertex
$v$. By defining \[ \operatorname{V}(P)\vert_n = \{ v \in \operatorname{V}(P) : \ell(v) = x \} \] to
be the set of vertices with label $x$, we can partition the pattern vertices by label into disjoint
sets $\{\operatorname{V}(P)\vert_1, \ldots, \operatorname{V}(P)\vert_k \}\textnormal{\,,}$ each of
which is expected to contain $\nicefrac{p}{k}$ vertices. Similarly, we may partition the target
vertices into disjoint sets $\{ \operatorname{V}(T)\vert_1, \ldots, \operatorname{V}(T)\vert_k \}$.

Without labels, there are $t^{\underline{p}} = t \cdot (t - 1) \cdot \ldots \cdot (t - p + 1)$
possible injective assignments of target vertices to pattern vertices.  With labels, observe that
for any label $x$, vertices in $\operatorname{V}(P)|_x$ may only be mapped to vertices in
$\operatorname{V}(T)|_x$.  Thus for each label $x$, we have an expected $\nicefrac{p}{k}$ variables,
each of whose domains contain $\nicefrac{t}{k}$ values. We would like to say that the size of the
state space is now \[ |S| = \left((\nicefrac{t}{k})^{\underline{\nicefrac{p}{k}}}\right)^{k}
\textnormal{,} \] but to do this we must state what $a^{\underline{b}}$ means when $b$ is
fractional. The gamma function $\Gamma(n)$ is equal to $(n - 1)!$ for integers $n \ge 1$
\citep{Davis59}, but is also defined for positive real numbers, obeying the identify $\Gamma(x + 1)
= x\Gamma(x)$. By noting that $t^{\underline{p}} = \nicefrac{t!}{(t - p)!}$, we may obtain a
reasonable continuous extension by taking \[ |S| = \left(\frac{\Gamma\left(\nicefrac{t}{k} +
1\right)}{\Gamma\left(\nicefrac{t}{k} - \nicefrac{p}{k} + 1\right)}\right)^{k} \textnormal{\,.} \]

As before, we expect the pattern to have $d_p \cdot \binom{p}{2}$ edges, and so if we simplify by
assuming the pattern will have \emph{exactly} this many edges, we obtain the
probability of all of these edges being mapped to edges in the target by raising $d_t$ to this
power, giving an estimate of \begin{equation}\label{equation:non-induced-labelled-prediction} \langle Sol \rangle = \left(
    \frac{\Gamma\left(\nicefrac{t}{k} + 1\right)}{\Gamma\left(\nicefrac{t}{k} - \nicefrac{p}{k} +
1\right)} \right)^{k}  \cdot
{d_t}^{d_p \cdot \binom{p}{2}} \textnormal{\,.} \end{equation}

\begin{figure}[t]
    \centering
    \begin{tikzpicture}[every node/.style={inner sep=0pt, outer sep=0pt}]
        \matrix {
            \node [anchor=center] {}; &
            \node [anchor=center] {\scriptsize No labels\hspace*{1em}}; &
            \node [anchor=center] {\scriptsize 2 labels\hspace*{1em}}; &
            \node [anchor=center] {\scriptsize 3 labels\hspace*{1em}}; &
            \node [anchor=center] {\scriptsize 5 labels\hspace*{1em}}; &
            \node [anchor=center] {\scriptsize 10 labels\hspace*{1em}}; &
            \node [anchor=center] {\scriptsize 20 labels\hspace*{2em}}; &
            \\
            \node [anchor=center, inner sep=2pt, rotate=90] {\scriptsize Satisfiable?}; &
            \node [anchor=center] {\includegraphics*{plots/graph-non-induced-satisfiable-20-150.pdf}}; &
            \node [anchor=center] {\includegraphics*{plots/graph-non-induced-satisfiable-20-l2-150.pdf}}; &
            \node [anchor=center] {\includegraphics*{plots/graph-non-induced-satisfiable-20-l3-150.pdf}}; &
            \node [anchor=center] {\includegraphics*{plots/graph-non-induced-satisfiable-20-l5-150.pdf}}; &
            \node [anchor=center] {\includegraphics*{plots/graph-non-induced-satisfiable-20-l10-150.pdf}}; &
            \node [anchor=center] {\includegraphics*{plots/graph-non-induced-satisfiable-20-l20-150.pdf}}; &
            \\
            \node [anchor=center, inner sep=2pt, rotate=90] {\scriptsize Glasgow}; &
            \node [anchor=center] {\includegraphics*{plots/graph-non-induced-nodes-20-150.pdf}}; &
            \node [anchor=center] {\includegraphics*{plots/graph-non-induced-nodes-20-l2-150.pdf}}; &
            \node [anchor=center] {\includegraphics*{plots/graph-non-induced-nodes-20-l3-150.pdf}}; &
            \node [anchor=center] {\includegraphics*{plots/graph-non-induced-nodes-20-l5-150.pdf}}; &
            \node [anchor=center] {\includegraphics*{plots/graph-non-induced-nodes-20-l10-150.pdf}}; &
            \node [anchor=center] {\includegraphics*{plots/graph-non-induced-nodes-20-l20-150.pdf}}; &
            \\
            \node [anchor=center, inner sep=2pt, rotate=90] {\scriptsize LAD}; &
            \node [anchor=center] {\includegraphics*{plots/graph-lad-non-induced-nodes-20-150.pdf}}; &
            \node [anchor=center] {\includegraphics*{plots/graph-lad-non-induced-nodes-20-l2-150.pdf}}; &
            \node [anchor=center] {\includegraphics*{plots/graph-lad-non-induced-nodes-20-l3-150.pdf}}; &
            \node [anchor=center] {\includegraphics*{plots/graph-lad-non-induced-nodes-20-l5-150.pdf}}; &
            \node [anchor=center] {\includegraphics*{plots/graph-lad-non-induced-nodes-20-l10-150.pdf}}; &
            \node [anchor=center] {\includegraphics*{plots/graph-lad-non-induced-nodes-20-l20-150.pdf}}; &
            \\
            \node [anchor=center, inner sep=2pt, rotate=90] {\scriptsize VF2}; &
            \node [anchor=center] {\includegraphics*{plots/graph-vf2-non-induced-nodes-20-150.pdf}}; &
            \node [anchor=center] {\includegraphics*{plots/graph-vf2-non-induced-nodes-20-l2-150.pdf}}; &
            \node [anchor=center] {\includegraphics*{plots/graph-vf2-non-induced-nodes-20-l3-150.pdf}}; &
            \node [anchor=center] {\includegraphics*{plots/graph-vf2-non-induced-nodes-20-l5-150.pdf}}; &
            \node [anchor=center] {\includegraphics*{plots/graph-vf2-non-induced-nodes-20-l10-150.pdf}}; &
            \node [anchor=center] {\includegraphics*{plots/graph-vf2-non-induced-nodes-20-l20-150.pdf}}; &
            \\
            % \node [anchor=center, inner sep=2pt, rotate=90] {\scriptsize VF2 Failures}; &
            % \node [anchor=center] {\includegraphics*{plots/graph-vf2-non-induced-abortions-20-150.pdf}}; &
            % \node [anchor=center] {\includegraphics*{plots/graph-vf2-non-induced-abortions-20-l2-150.pdf}}; &
            % \node [anchor=center] {\includegraphics*{plots/graph-vf2-non-induced-abortions-20-l3-150.pdf}}; &
            % \node [anchor=center] {\includegraphics*{plots/graph-vf2-non-induced-abortions-20-l5-150.pdf}}; &
            % \node [anchor=center] {\includegraphics*{plots/graph-vf2-non-induced-abortions-20-l10-150.pdf}}; &
            % \node [anchor=center] {\includegraphics*{plots/graph-vf2-non-induced-abortions-20-l20-150.pdf}}; &
            % \\
        };
    \end{tikzpicture}
    \caption{On the top row, predicted and actual location of the phase transition for labelled
    non-induced random subgraph isomorphism, with a pattern order of 20, a target order of 150,
    varying pattern (x-axis) and target (y-axis) density, and varying numbers of labels. On
    subsequent rows, the average number of search nodes needed to solve an instance, for three
    different solvers}\label{figure:labels}
\end{figure}

So how good are these predictions? The black lines on the first row of heatmaps in
\cref{figure:labels} plot where we calculate $\langle Sol \rangle = 1$ will occur. For small numbers
of labels, our predictions are slightly better than in the unlabelled case: there seems to be less
of a variance problem for very dense patterns. Even as the number of labels becomes relatively
large, the prediction of the phase transition still occurs in the right place, but with ten labels
we start to see sporadic unsatisfiable instances deep inside the satisfiable region. With twenty
labels, we instead get a kind of phase transition from ``all unsatisfiable'' to ``mixed satisfiable
and unsatisfiable''. We can understand this intuitively: with sufficiently many labels, we might
generate, say, a red vertex adjacent to a blue vertex in the pattern, but not in the target.  With
twenty labels, we even sometimes generate no vertices with a particular label in the target at all.
This in many ways resembles ``flaws'' generated by certain random constraint satisfaction problem
instance generators
\citep{DBLP:journals/constraints/AchlioptasMKSKK01,DBLP:journals/constraints/GentMPSW01}.

What about empirical hardness? As before, some instances on the phase transition are hard for all
solvers (although as the number of labels increases, the hardest instances become easier).
Instances far from the phase transition are easy, except for VF2, which occasionally finds
some of the instances with larger numbers of labels very difficult. This behaviour occurs both on
satisfiable instances, and on the flawed unsatisfiable instances deep inside the ``satisfiable''
region.  It is interesting to observe that all such unsatisfiable instances that VF2 finds difficult
have a very small proof of unsatisfiability, with neither Glasgow nor LAD requiring more than one
hundred recursive calls.

\subsection{Richer Label Models}\label{subsection:richer-labels}

Why does VF2 find some of these instances so hard? Unlike Glasgow and LAD, VF2 does not track
domains, and so cannot detect that there are no suitable target vertices available for a given
pattern vertex until it branches on that vertex.  It also cannot detect small domains, and only uses
``adjacency to an existing assignment'' as a branching heuristic. This can make it very hard for VF2
to detect that it is in an obviously failed state, or that an instance or subproblem is trivially
unsatisfiable.

\begin{figure}[t]
    \centering
    \includegraphics*{plots/skewed.pdf}

    \caption{The cumulative number of instances solved over time, using the richer model of
    randomness described in \cref{subsection:richer-labels}.}
    \label{figure:skewed-cumulative}
\end{figure}

To illustrate this point further, we now look at some instances created using a slightly more
structured random model. We create a family of one thousand labelled instance pairs, as follows. To
create a pattern graph, we create ten vertices with label zero, with edges between these vertices
with probability 0.2. We then add another ten vertices, with labels chosen randomly between one and
thirty, and add edges between these vertices and the zero-labelled vertices with probability 0.1.
To generate a target, we follow a similar process: we have fifty vertices with label zero and edge
probability 0.2, fifty vertices with labels randomly between one and thirty, and edges from the
first set of vertices to the second set with probability 0.3.  This model was selected and the
parameters tuned to provide a demonstration of particular behaviours of VF2, not because of any natural
property (although inspiration came from seeking a very crude approximation of chemical graphs,
which often contain a lot of carbon in the centre, and other atoms around the outside). From our set
of one thousand such instances, sixty six are satisfiable.

We plot the cumulative number of instances solved over time for these instances in
\cref{figure:skewed-cumulative}: for a given runtime choice along the $x$-axis, the $y$ value for a
given algorithm shows how many instances, individually, took at most $x$ milliseconds to solve with
that algorithm. Both Glasgow and LAD find all of these instances trivial, with no instance requiring
even ten milliseconds to solve. VF2, in contrast, finds many of them extremely difficult, and cannot
solve 352 of the 1,000 instances with a one thousand second timeout.

The ``VF2${\leftrightarrow}$'' line shows what happens with VF2 if the graphs are permuted, so that
the non-zero labelled vertices are given lower vertex numbers rather than higher vertex numbers.
In other words, we permute pattern vertices in such a way that the first ten vertices have non-zero
labels, and the ten last vertices have zero labels (and similarly for the target graphs).
This makes VF2 behave better, but still extremely poorly compared to the other two solvers; the
motivation behind this modification will become clear later in this paper.  It will also be
important to remember that both VF2 variations find some satisfiable \emph{and} some unsatisfiable
instances extremely hard.

An even more extreme example of VF2's misbehaviour can be seen in \cref{figure:vf2-stupidity}. Here
we have a pattern graph and a target graph, both of which are cliques plus one isolated vertex, and
the isolated vertices have different unique labels. This is trivially unsatisfiable (and Glasgow and
LAD detect this without search), but unless the labelled vertex is given the lowest vertex number
(so it is branched on first), VF2 takes nearly a million recursive calls to detect this: VF2 will
try to map every adjacent vertex in the clique before considering the unmatchable isolated vertex.
This example can be extended slightly to fool any simple static heuristic, or any simple label
counting mechanism (for example, by attaching an additional vertex with a unique label to the
clique). Note also the resemblance to the cases seen in the bottom left of the VF2 plots in
\cref{figure:non-induced}, where a pattern graph which has a few edges combined with a target graph
with no edges gives a timeout.

These experiments further highlight that VF2 occasionally finds some instances which should be easy hard.
But does this cause problems in practice? The literature suggests that it does. For example,
\citet{o:Gromping14} uses VF2 in a package for the R statistics language, and states
\begin{displayquote}``There are (not so many) instances for which creation of a clear design is
    prohibitively slow in the current implementation that evaluates subgraph isomorphism with the
    VF2 algorithm \ldots Recent experiences with a few of these showed that the LAD algorithm was
    very fast in ruling out impossible matches, where VF2 took a long time.''\end{displayquote}
Similarly, \citet{o:Murray12} uses VF2 inside a compiler, and observes extremely variable
(and prohibitively high) compile times in some cases due to the expense of subgraph isomorphism
calls---given the heavily labelled nature of these graphs, we conjecture that any domain-based
algorithm would eliminate this cost.

\begin{figure}[tb]
    \centering
    \begin{tikzpicture}[scale=0.5,every node/.style={font=\footnotesize}]
        \begin{scope}
            \newcount \myc
            \newcount \myd
            \newcount \mye
            \foreach \n in {1, ..., 9}{
                \myc=\n \advance\myc by -1 \multiply\myc by -360 \divide\myc by 10 \advance\myc by 18.0
                \node[draw, circle, fill=chromaygb1, inner sep=0.5pt] (N\n) at (\the\myc:2.4) {\phantom{0}};
            }
            \node[draw, circle, fill=chromaygb3, inner sep=0.5pt] (N10) at (54.0:2.4) {\phantom{0}};

            \foreach \n in {1, ..., 8}{
                \myd=\n
                \myc=\n \advance\myc by 1
                \foreach \m in {\the\myc, ..., 9}{
                    \mye=\m
                    \draw (N\the\myd) -- (N\the\mye);
                }
            }
        \end{scope}
        \begin{scope}[xshift=9cm]
            \newcount \myc
            \newcount \myd
            \newcount \mye
            \foreach \n in {1, ..., 9}{
                \myc=\n \advance\myc by -1 \multiply\myc by -360 \divide\myc by 10 \advance\myc by 18.0
                \node[draw, circle, fill=chromaygb1, inner sep=0.5pt] (N\n) at (\the\myc:2.4) {\phantom{0}};
            }
            \node[draw, circle, fill=chromaygb7, inner sep=0.5pt] (N10) at (54.0:2.4) {\phantom{0}};

            \foreach \n in {1, ..., 8}{
                \myd=\n
                \myc=\n \advance\myc by 1
                \foreach \m in {\the\myc, ..., 9}{
                    \mye=\m
                    \draw (N\the\myd) -- (N\the\mye);
                }
            }
        \end{scope}
    \end{tikzpicture}
    \caption{A trivially unsatisfiable subgraph isomorphism instance which VF2 finds exponentially
    difficult. The pattern (left) and target (right) both consist of a clique, plus one extra vertex
    which has a different label in each graph.}\label{figure:vf2-stupidity}
\end{figure}


However, by far the biggest problem is in graph databases. The following section shows how
widespread use of poor subgraph isomorphism algorithms has not just lead to overly pessimistic
conclusions regarding performance, but has misdirected the design of larger systems.

\section{Querying Graph Databases}\label{section:filterverify}

A particularly common use of subgraph isomorphism algorithms is inside graph databases.  The
general problem these systems solve is, for a set of target graphs, to process a pattern query and
return every target graph which is subgraph-isomorphic to that pattern. The set of target graphs is
usually seen as fixed, or at least rarely-changing, whilst the patterns arrive dynamically. This has
lead to the development of systems which perform extensive computations on the target graphs
beforehand, in the hopes of reducing the response times for individual pattern queries. The most
popular of these strategies is a form of indexing which is often named \emph{filter~/ verify}.

\subsection{The Filter / Verify Paradigm}

The filter~/ verify approach has an interesting history, of which we now give a very selective and
incomplete overview. Our description is biased by a general modern understanding of the empirical
hardness of \NP-complete problems, which was not widely known at the time of the earlier
papers we discuss. The common theme of all of the following papers is that pre-computed information
is used to eliminate certain unsatisfiable instances from consideration, without performing a
subgraph isomorphism test. For example, an index might contain a bit of information expressing
whether a target graph contains at least two red vertices. When a pattern graph with two red
vertices is used as a query, any target whose feature set does not have this bit set would not be
considered, and so a subgraph isomorphism call would not be made for that pattern / target pair. (In
practice, feature hashing is often used, which can lead to false positives, but this is not relevant
to our discussion.)

An early graph database system by \citet{DBLP:conf/pods/ShashaWG02} uses a filtering heuristic to
eliminate unsatisfiable instances based upon simple structural elements. It is not clear whether the
aim is to reduce I/O costs, or to reduce the number of queries which must be tested, and the
experiments do not answer whether the indexing is effective. However, the work was influenced by a commercial
graph database system, whose documentation \citep{o:Daylight} states that indexing is
used to minimise disk accesses.

Subsequently, in a widely cited paper, \citet{DBLP:conf/sigmod/YanYH04} introduce an indexing system
called gIndex. Again, this system handles queries by first producing a set of candidates by
eliminating certain unsatisfiable instances, this time by using substructures. They argue that the
query response time, which is to be minimised, is governed by the equation \[ T =
T_{\mathit{search}} + \left|C_q\right| \cdot T_{\mathit{iso\_test}} \textnormal{,}\] where
$T_{\mathit{search}}$ is the time taken to search for a candidate set of potential solutions of size
$\left|C_q\right|$, and $T_{\mathit{iso\_test}}$ is the cost of a subgraph isomorphism test. They
reason that since isomorphism testing is \NP-complete, by making $\left|C_q\right|$ as small as
possible, the query response time will be reduced.  (Importantly, it is \emph{not} the time taken to
load graphs from disk which contributes to the per-candidate cost, but rather the time to perform a
subgraph isomorphism call.) They conclude that ``graph indexing plays a critical role at efficient
query processing in graph databases''.

This equation is repeated and expanded upon by \citet{DBLP:journals/tods/YanYH05} to explicitly
separate I/O and isomorphism testing costs, obtaining a query response time of \[
    T_{\mathit{search}} + \left|C_q\right| \cdot (T_{\mathit{io}} + T_{\mathit{iso\_test}})
    \textnormal{.} \] The authors explicitly state that ``the value of $T_{\mathit{iso\_test}}$ does
not change much for a given query''. The argument presented is that
\begin{displayquote}``Sequential scan is very costly
because one has to not only access the whole graph database but also check subgraph isomorphism. It
is known that subgraph isomorphism is an \NP-complete problem. Clearly, it is necessary to build
graph indices in order to help processing graph queries.''\end{displayquote}
With what we now know about the behaviour of modern subgraph isomorphism algorithms, and the nature
of solving \NP-complete problems in general, we should immediately be suspicious of these claims. We
do not expect $T_{\mathit{iso\_test}}$ to be anything like a constant, even if the orders of the
input graphs are similar. In particular, any instance which can be excluded based upon filtering
must have a very small proof of unsatisfiability. These instances \emph{should} be trivial with any
decent subgraph isomorphism algorithm. Thus, all filtering \emph{should} be doing is eliminating the startup
costs of a trivial subgraph isomorphism call. The fact that filtering was successful empirically
should make us wonder whether the subgraph isomorphism algorithm being used was excessively
primitive. Indeed, the subgraph isomorphism algorithm used is described only as ``the simplest approach'' in the
paper.  Additionally, the experiments focus on reducing the size of the candidate set, without
considering the time taken to verify different candidate set instances.  The claim that
$T_{\mathit{iso\_test}}$ does not change much is not justified experimentally, and no consideration
is given as to whether this would hold true for other subgraph isomorphism algorithms.

Moving forwards, the (simpler form of the) query response time equation is repeated by
\citet{DBLP:conf/vldb/ZhaoYY07}. Again, the work has a focus on reducing the candidate set size
through indexing. The authors appear to believe that the cost of the isomorphism test is not a major
factor in influencing the result, and focus on the remaining terms in the equation. They use the
``average cost'' of a subgraph isomorphism test as a constant, without considering that the average
cost could be influenced by the character of the candidate set.

The equation is also used by \citet{DBLP:conf/icde/JiangWYZ07}, who claim that ``usually the
verification time dominates the Query Response Time [s]ince the computational complexity of
$T_{\mathit{iso\_test}}$ is NP-Complete''. They note that \begin{displayquote}``Approximately, the
    value of $T_{\mathit{iso\_test}}$ does not change too much with the difference of query. Thus,
the key to reducing query response time is to minimize the size of the candidate answer
set''.\end{displayquote} This claim becomes understandable when one examines the subgraph
isomorphism algorithm chosen for the verification step \citep{DBLP:journals/jacm/Ullmann76}: as the
algorithm predates techniques like generalised arc-consistent all-different propagation
\citep{DBLP:conf/aaai/Regin94}, it cannot immediately detect unsatisfiability in simple cases like
there being two red vertices in the pattern but only one in the target\footnote{Although
interestingly, this algorithm effectively does forward-checking and has a variable-ordering
heuristic, before these concepts appeared in the constraints literature.}.

Without using the equation, \citet{DBLP:conf/sigmod/ChengKNL07} state that since subgraph
isomorphism is NP-complete, processing by a sequential scan is infeasible. They introduce new
filtering techniques to try to avoid the subgraph isomorphism step. A similar claim is made by
\citet{DBLP:conf/icde/ZhangHY07} in an introduction of another indexing technique: ``obviously it is
inefficient to perform a sequential scan on every graph in the database, because the subgraph
isomorphism test is expensive''.

Muddying the waters slightly, a survey by \citet{DBLP:journals/datamine/HanCXY07} states that
``large volumes of data'' (not \NP-completeness) is the reason for using indexing in these systems.
However, in a description of a system tailored to biological networks,
\citet{DBLP:conf/edbt/ZhangLY09} state that \begin{displayquote}``Since the size of the raw database
    graph is small, it can be easily fit in the main memory.  However, the query (matching) time
will be very long due to the NP-hard complexity.''\end{displayquote} They suggest that indexing is
a way of avoiding this. Similarly, \citet{DBLP:journals/pvldb/ZhaoH10} argue that ``the graph query
problem is hard in that \ldots it requires subgraph isomorphism checking \ldots which has proven to
be NP-complete'' and that working with large networks is hard or impossible ``due to the lack of
scalable graph indexing mechanisms''.

\citet{DBLP:conf/icdcs/CaoYWRL11} propose a privacy-preserving cloud graph database system using
filter~/ verify, stating that ``checking subgraph isomorphism is \NP-complete, and therefore it is
infeasible to employ such a costly solution'' which simply checks every graph for a match.
\citet{DBLP:journals/tkde/WangWYY12} look at indexing large sparse graphs. They state that ``because
subgraph isomorphism is an \NP-complete problem, a filter-and-verification method is usually
employed to speed up the search efficiency of graph similarity matching over a graph set''.
Similarly, after reviewing the literature, \citet{DBLP:journals/vldb/YuanM13} argue that subgraph
querying is costly because it is \NP-complete, and that indices can improve the performance of graph
database queries. Again, new indexing techniques are introduced. Subsequently,
\citet{DBLP:journals/pvldb/KatsarouNT15} perform a comprehensive comparison of graph database
filtering techniques. They state that performing a query against each graph in the dataset
``obviously does not scale, as subgraph isomorphism is \NP-complete''. In describing a system which
returns a special subset of graphs which match a query, \citet{DBLP:journals/tkde/ZhengLZHZ16}
suggest that it is ``NP-hard to check the graph isomorphism'' (meaning subgraph isomorphism), and
``in order to improve the time efficiency'' they use an indexing system to ``avoid as many costly
subgraph isomorphism checkings as possible''. Their index takes tens of thousands of seconds to
build, and they suggest that Ullmann's algorithm and VF2 are state of the art for verification.
\citet{DBLP:journals/www/PengZ0LZ16} state that ``obviously, it is impossible to employ some
subgraph isomorphism algorithm, such as Ullmann or VF2'', and argue that ``in order to speed up
query processing'', they need to create indices.  And \citet{DBLP:conf/isda/AzaouziR16} argue that
``since the subgraph isomorphism test is expensive, checking all graphs of a large database can be
unfeasible'', saying that ``naturally, the verification step is computationally more expensive since
it requires a subgraph isomorphism''; their experiments look at candidate set reduction sizes, and
their choice of subgraph isomorphism algorithm is not mentioned.

The filter~/ verify paradigm also influences other research. For example, it is used by
\citet{DBLP:conf/icde/TianP08} in an approximate subgraph searching system: they state that
Ullmann's algorithm ``is prohibitively expensive for querying against [a] database with a large
number of graphs'', and that indices are used ``to filter out graphs that do not match the query''.
More recently, \citet{DBLP:journals/pvldb/YuanMG13} continue a line of supergraph search work, again
using a filtering step to avoid subgraph isomorphism calls (in the opposite direction, so queries
are now target graphs). \citet{DBLP:journals/tkde/HongZLY15} look at graph database subgraph
isomorphism with an additional set-similarity constraint, and state that Ullmann's algorithm and VF2
are ``usually costly for large graphs'' because they ``do not utilize any index structure''. They
propose an indexing structure, which takes over 2,000 seconds to construct, and uses nearly 2GBytes
of space.  There is also research into maintaining indices when the set of target graph changes: for
example \citet{DBLP:conf/sigmod/YuanMYG15} look at algorithms for updating graph indices.
And describing a system for reusing results of queries which are sub- or super-graphs of previous
queries, \citet{DBLP:conf/edbt/WangNT16} state that querying is a ``very costly operation as it
entails the \NP-complete problem of subgraph isomorphism'', and place ``an emphasis on the number of
unnecessary subgraph isomorphism tests''.

After some early ambiguity, then, it becomes clear that the intent behind filter~/ verify systems is
to reduce the number of subgraph isomorphism calls, and that the cost of loading graphs from disk is
not considered to be problematic. It is worth noting that the entire test datasets from most of
these papers will comfortably fit in RAM on a modern desktop machine, even when an adjacency matrix
representation is used.

Thus we can see that there are two critical beliefs underlying all of this work---firstly, that
subgraph isomorphism is necessarily hard because it is \NP-complete, and secondly, that there are
ways of identifying unsatisfiable instances using short proofs that a subgraph isomorphism algorithm
will not detect, but that an indexing system can. Throughout, the cost models used assume that the
time for a subgraph isomorphism query does not particularly depend upon the instance, and nowhere is
it considered that a good subgraph isomorphism algorithm should be able to eliminate
obviously-unsatisfiable instances with a similar time requirement to an indexing system.

These beliefs are not entirely unfounded: none of the isomorphism algorithms considered in these
papers will immediately detect if a pattern contains two red vertices, whilst the target graph
contains only one. This kind of flaw \emph{should} be picked up at the top of search by an
all-different propagator \citep{DBLP:conf/aaai/Regin94}.  However, as
\citet{DBLP:journals/pvldb/KatsarouNT15} note, VF2 \citep{DBLP:journals/pami/CordellaFSV04} or a
similar algorithm is the usual subgraph isomorphism algorithm of choice for graph database systems,
although \citeauthor{DBLP:journals/jacm/Ullmann76}'s algorithm is sometimes chosen. Other approaches
have been considered, albeit not with algorithms strong enough to establish all-difference. For
example, \citet{DBLP:journals/pvldb/ShangZLY08} propose an algorithm which makes use of the
frequency of various features to guide search; \citet{DBLP:journals/pvldb/LeeHKL12} determine
experimentally that this technique tends to be very effective, even on families of graphs for which
it was not designed.

We therefore believe it would be unlikely to cause too much astonishment if we suggested that a
better subgraph isomorphism algorithm could be dropped in as a black box replacement in graph
databases systems to improve their performance.  This is not our claim. Instead, this paper shows
that better algorithms invalidate the flawed premise underlying the entire filter~/ verify approach.
We will now demonstrate empirically that filter~/ verify is simply a poor workaround for the kinds
of deficiency in VF2 demonstrated towards the end of the previous section.

\subsection{Is Filtering Necessary?}

To show that a pure subgraph isomorphism approach is feasible, with no indexing or supporting
preprocessing, we look at four datasets commonly used in graph indexing evaluations. We do not claim
that these are high-quality datasets or that the associated queries are sensible, merely that
following the example of \citet{o:10.1371/journal.pone.0076911}, they are widely used.

\begin{itemize}
    \item The AIDS dataset contains graphs representing 40,000 chemical molecules. These graphs are
        labelled, and are fairly small (mean 45 vertices) and sparse. Following
        \citeauthor{o:10.1371/journal.pone.0076911}, the queries are compounds with 8, 16 or 32
        edges.
    \item The PDBS dataset \citep{o:HeLCBBSKMR02} contains 30 labelled graphs representing DNA, RNA,
        and proteins, each duplicated twenty times to give a dataset of 600 graphs. These
        can be relatively large, having up to tens of thousands of vertices, but are extremely
        sparse. The queries are randomly selected connected subgraphs and do not have a real-world
        meaning.
    \item The PCM dataset \citep{o:doi:10.1093/bioinformatics/btr163} contains 50 protein contact
        maps, each duplicated four times to give a dataset of 200 graphs. These graphs have under a
        thousand vertices and below twenty thousand edges; they are slightly less sparse than the
        previous two datasets. As for PDBS, the queries are randomly generated and do not have a
        particular meaning.
    \item The PPI dataset contains 20 protein interaction networks, with up to ten thousand
        vertices. The queries have either four or eight vertices.
\end{itemize}

\begin{figure}[t]
    \centering
    \hspace*{1em}
    \includegraphics*{plots/biiiig-data-aids.pdf}
    \hfill
    \includegraphics*{plots/biiiig-data-pcms.pdf}
    \hspace*{1em}

    \vspace*{1em}

    \centering
    \hspace*{1em}
    \includegraphics*{plots/biiiig-data-pdbs.pdf}
    \hfill
    \includegraphics*{plots/biiiig-data-ppigo.pdf}
    \hspace*{1em}

    \caption{Cumulative number of instances solved as a function of search space size, using four
    graph database datasets and a simple constraint programming subgraph isomorphism algorithm. Note
    that the $x$-axis shows recursive calls, and does \emph{not} use a log scale.}
    \label{figure:biiiig-data}
\end{figure}

Rather than the modern techniques used by Glasgow and LAD, we deliberately select a very simple
starting point: a constraint programming model implemented in Gecode 5.1.0 using only
toolkit-provided constraints and heuristics. The model uses a ``smallest domain first'' variable
ordering heuristic with tie-breaking on degree, an all-different constraint, extensional constraints
for adjacency, and labelled-degree filtering at the top of search. Even this simple approach finds
nearly every instance in each of the four datasets trivial, and no exponential behaviour is
observed. As we show in \cref{figure:biiiig-data}, the \emph{hardest} instance for the AIDS dataset
requires 287 recursive calls, and 149,268 of the instances can be solved without search.  For PDBS,
the hardest problem requires 3,161 recursive calls (which occurs 20 times, due to duplicated
queries). For both PCM and PPI, the hardest problem requires 23 recursive calls.  In other words,
none of these instances are in any way computationally hard even for a simple domain-based
algorithm, even before introducing stronger filtering, inference, or heuristics.  Furthermore it is
certainly not the case that a sequential scan is infeasible as so many filter~/ verify papers claim.
The exponential behaviour seen by others in these instances is purely down to a bad choice of
subgraph isomorphism algorithm, and these instances are not inherently ``really hard''.

\subsection{Benefits of Constraint Programming over Filter / Verify}

There are at least four benefits beyond simplicity towards abolishing filter~/ verify in favour of a
purely constraint programming inspired approach:

\begin{itemize}
    \item Domain filtering is useful on both satisfiable and unsatisfiable instances, whilst
        filtering can only eliminate trivially unsatisfiable instances, and is entirely wasted on
        satisfiable instances. Domain filtering is also useful even on relatively hard instances,
        since the information cuts down the search space rather than providing a simple ``yes'' or
        ``no''.

    \item Domain-based heuristics are much stronger than VF2's adjacency branching rules. Picking
        from small domains dynamically allows search to focus on the hardest part of the problem.

    \item This approach automatically combines features. For example, a pattern~/ target pair may
        have matching label features, and matching degree features, but if the only red pattern
        vertex has degree three whilst no red target vertex has degree more than two, domain
        filtering will detect this immediately. When combined with all-different propagation and
        maintained during search, this effect is even stronger.

    \item Finally, as indexing systems get more and more complex in an attempt to filter out a few more
        instances where VF2 performs poorly, the cost of index construction and maintenance is
        considerable.  Indices proposed in the literature often take many hours to build, and can
        consume much more space than the original graphs.
\end{itemize}

In other words, using domain-based algorithms would not simply be a viable alternative to filter~/
verify with VF2, but rather would be a much better solution.

\subsection{Other Implications}

Unfortunately, indexing is not the only incorrect design choice being made in systems built around
subgraph isomorphism algorithms, and lessons from constraint programming can be applied to other
subgraph-related research. We illustrate this by looking at several recent papers on subgraph
isomorphism, discussing how a deeper understanding of the empirical hardness of subgraph isomorphism
could lead the reader to different conclusions than the ones presented.

\citet{DBLP:conf/edbt/KatsarouNT17} look at what they call ``straggler'' queries, which they define
to be the small subset of queries that they observe taking much longer than others to solve. The
authors present an approach to address this perceived problem that they describe as novel: ``instead
of trying to come up with new algorithms for sub-iso testing, we utilize isomorphic query rewritings
and existing alternative algorithms in parallel''. Their first claim is that permuting graphs before
running the subgraph isomorphism algorithm can lead to ``wildly different'' execution times. They
note that VF2 does not define ``a strict order in which the nodes of the query are matched'', and so
permuting the graphs (for example, by using degree or label frequency information) before search can
sometimes give orders of magnitude improvements. The connection to variable- and value-ordering
heuristics is not noted, and no consideration is given to even the simplest dynamic ordering
heuristics like ``smallest domain first'' \citep{DBLP:journals/ai/HaralickE80}. A little thought
shows that of the orderings proposed, those involving placing rare labels first are effectively poor
approximations to a static ``smallest domain at top of search first'' ordering, whilst the remainder
use degree as we did earlier in this paper. However, \citeauthor{DBLP:conf/edbt/KatsarouNT17} do not
investigate any algorithm which employs domains, let alone strong variable- or value-ordering
heuristics or all-different propagation, and do not consider that their apparent successes could be
due to VF2's weaknesses rather than an inherent property of \NP-completeness.

We saw in the previous sections that although permuting input graphs could improve VF2's behaviour
on some instances, doing so does not make its performance come close to that of domain-based
algorithms.  \citeauthor{DBLP:conf/edbt/KatsarouNT17} say they ``hope that our findings will open up
new research directions, striving to find appropriate, per-query, isomorphic rewritings, in
combination with alternate per-query sub-iso algorithms that can yield large improvements''. We
believe it is important instead to emphasis previous research directions that have already solved
most of this problem, and to help ensure that these techniques become more widely known.

\citet{DBLP:conf/edbt/KatsarouNT17}'s second claim is that ``different
algorithms find different queries hard''. To address this, they run many subgraph isomorphism
algorithms and input permutations in parallel; the extensive literature on parallel portfolios in
general \citep{DBLP:journals/ai/GomesS01}, and the approach by \citet{DBLP:conf/sls/BattitiM07} for
subgraph isomorphism in particular, is not noted (and nor are portfolios mentioned when they say
that ``using machine learning models to predict which version \ldots to employ per query is of high
interest'').  The evidence so far in this paper suggests that we should be wary. It is certainly
true that there are some instances that all algorithms find hard, and there are good theoretical
reasons to believe that these instances genuinely are really hard.  Furthermore, algorithm
portfolios are also well-known to be a successful technique, and are beneficial even with modern
subgraph isomorphism algorithms \citep{DBLP:conf/lion/KotthoffMS16}.  However, in
\cref{section:labelled} we saw that there are many instances that VF2 finds hard that should not be
hard, and that are not hard for other algorithms.  If permuting graphs sometimes addresses the
difficulty of some of these instances, then it is likely that they are not genuinely hard instances
at all, and perhaps it would be better simply to start using an algorithm with domain tracking and
domain-based ordering heuristics, rather than assembling a portfolio of bad algorithms in the hopes
that at least one of them will often get lucky.

To test this suggestion, we return briefly to the experiments on the datasets discussed earlier in
this section. What if we modify our solver so that it does not use ``smallest domain first'', and
does not detect domain wipeouts until attempting to branch on an empty variable?  Suddenly three of
the four datasets become extremely hard, with many instances now timing out after making hundreds of
millions of recursive calls, and the PPI dataset now requires hundreds of thousands of calls for
some queries.  Furthermore, slight changes to the input vertex ordering can now make some of these
``hard'' instances easy again.  This backs up our suspicions: the apparent success of
\citet{DBLP:conf/edbt/KatsarouNT17}'s technique is due to the ease of sometimes getting better
results out of a poor algorithm. Claiming huge improvements from an algorithm portfolio consisting
only of variations of poor algorithms does not demonstrate a genuine improvement over the state of
the art.

Improved search orderings for VF2-style algorithms also appear in recent work by
\citet{DBLP:conf/gbrpr/CarlettiFV15} and \citet{o:Carletti16}, cumulating in VF3
\citep{DBLP:conf/gbrpr/CarlettiFSV17,o:CarlettiFSV17}, and by \citet{DBLP:conf/wise/ShenZ17}. We
have seen in \cref{section:labelled} that such an approach could give improvements, but will not
bring an algorithm that does not use domains close to the performance of one that does on harder
instances.

The importance of using hard benchmark instances was stressed by
\citet{DBLP:journals/heuristics/Gent98}:

\begin{displayquote}
    ``Benchmark problems should be hard. I report on the solution of the five open benchmark
    problems introduced \ldots for testing bin packing problems. Since the solutions were found
    either by hand or by using very simple heuristic methods, these problems would appear to be
    easy. In four cases I give improved packings to refute conjectures that previously reported
    packings were optimal, and I give a proof that the fifth conjecture was correct. \ldots Future
    experimenters should be careful to perform tests on problems that can reasonably be regarded as
    hard.''
\end{displayquote}

Unfortunately, this lesson has not sunk in. For example, \citet{DBLP:journals/tcbb/BonniciG17}
recently compared various non-constraint-based approaches to LAD using benchmark sets whose
difficulties are measured between microseconds and milliseconds. Their results suggest that LAD is
an order of magnitude slower than other approaches. This should not surprise us: the initialisation
costs of LAD are non-trivial due to it constructing domains, calculating neighbourhood degree
sequences, and performing all-different filtering. However, the risks of occasionally encountering
exponential performance from weaker algorithms on easy instances should now be clear.

\citet{o:CarlettiFSV17} correctly observe that ``a performance improvement on [hard instances] has a
far greater practical impact than it would have on easier graphs, since the saved time can be of
several hours or even days'', but mistakenly believe that being larger and slightly denser is what
makes an instance harder. To evaluate VF3, they introduce a new dataset consisting entirely of
satisfiable instances created by taking a connected subgraph of a randomly generated graph and
permuting the nodes. They conclude that VF3 is the best algorithm based upon its performance solving
the enumeration problem on these new instances.  We urge a more cautious interpretation of these
results: as we saw in \cref{section:induced}, being genuinely hard (rather than simply larger) is a
subtle property, and it is not safe to say that an instance is hard simply because several
minor variations of a particular kind of algorithm find it hard. More care must thus be taken to
avoid accidentally measuring which algorithm scales best to large, easy instances due to it using
the simplest inference routines. Indeed, we saw in \cref{figure:induced} that VF3 performs
\emph{worse} than VF2 on genuinely harder instances, and additionally behaves erratically in parts
of the parameter space that every other solver finds easy; finally, an induced analogue of
\cref{figure:labels} confirms that VF3 also still struggles with many obviously unsatisfiable
instances.

Of course, hard random instances of the kind we describe in
\cref{section:non-induced,section:induced} are also atypical, and should not form the sole basis for
benchmarking. In particular, distance-based filtering
\citep{DBLP:conf/cp/AudemardLMGP14,DBLP:conf/cp/McCreeshP15,DBLP:conf/lion/KotthoffMS16} has little
to no effect on these instances, but is extremely beneficial on other kinds of graph, including
those from real-world applications.  These graphs also have a low degree spread and very little
structure to exploit, which could mislead experimenters as to the value of certain kinds of
inference. \citet{o:GentW95} give an example of an exam timetabling graph which contains an
unexpected ten-vertex clique, which would be extremely unlikely in a randomly generated graph with a
similar order and density.

\section{Conclusion}

We have shown how to generate small but hard instances for the non-induced and induced subgraph
isomorphism problems, which will help offset the bias in existing datasets. For non-induced
isomorphisms, behaviour was as in many other hard problems, but for induced isomorphisms we
uncovered several interesting phenomena: there are hard instances far from a phase transition,
constrainedness predicts this, and existing general techniques for designing heuristics do not work
in certain portions of the parameter space.

When labels were introduced, we saw that VF2 finds some instances hard which other algorithms find
easy. We looked at this kind of instance in more detail, and argued that this was due to flaws in
VF2's design, rather than an interesting property of \NP-completeness. Although inevitably there
will be instances that every algorithm finds hard, we see no excuse for an algorithm to exhibit
exponential behaviour in the case when there are two red vertices in a pattern and only one in a
target.

Unfortunately, we saw that this aspect of VF2's performance has had practical consequences, most
notably being the misdesign of graph database systems. By not trying even the simplest constraint
programming techniques from the literature, and ignoring all that is known about the empirical
hardness of \NP-complete problems, members of the graph databases community have misled themselves
into believing that extensive research into supporting techniques is important, whereas really all
they are doing is working around some but not all of the defects in their choices of subgraph
isomorphism algorithms. With this new understanding of what does and does not make subgraph
isomorphism hard, it is time for a radical rethink of how graph database systems work.

We do not claim that the ultimate ``big data'' subgraph matching algorithm already exists:
on the contrary, there is likely to be plenty of room for future improvement now that we understand
the importance of getting algorithm design right.  For example, a major disadvantage of using
domains is the relatively expensive initialisation costs, which quickly add up when dealing with
large numbers of trivial instances. Employing a presolver is an obvious approach---and VF2 is
actually good in this role \citep{DBLP:conf/lion/KotthoffMS16}---but there are other possibilities.
For example, minimal or lazy forward-checking
\citep{DBLP:conf/ictai/DentM94,DBLP:conf/cp/BacchusG95,o:Dent96,DBLP:conf/ecai/LarrosaM98} avoids
constructing every domain upfront, although adopting this require alternatives to ``smallest domain
first'' and to all-different. Such an approach may also be beneficial for huge target graphs, where
having domains range over the entire target is impractical.

There is also scope for precalculating supporting information about target graphs. For example,
neighbourhood degree sequences and supplemental graphs could both be pre-computed and stored. The aim
here is to reduce the initialisation costs of a good subgraph isomorphism algorithm, and not to
provide indexing (although additionally using this information as an index may not hurt, if
initialisation is still costly).

We must stress that we are not simply concluding that graph database systems should use a better
subgraph isomorphism algorithm.  Instead, this paper has shown that such systems need to be designed
respecting our understanding of the empirical hardness of \NP-complete problems, that subgraph
isomorphism algorithms should not be treated as a black box, and that lessons learned in constraint
programming and artificial intelligence should not go unheeded in other domains.

\acks{The authors would like to thank Kitty Meeks and Craig Reilly for their comments, and Iva
Babukova for her help with the datasets. The heatmap schemes used in this paper were created with
the assistance of the ``Chroma.js Color Scale Helper'' by Gregor Aisch.}

\bibliography{paper}
\bibliographystyle{theapa}

\end{document}
